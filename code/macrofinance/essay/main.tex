\documentclass[american]{scrartcl}
    \usepackage{babel}
    \usepackage[utf8]{inputenc} 
    \usepackage{csquotes}
    \usepackage{amsmath}
    \usepackage{graphicx}    

    
    \setlength{\parindent}{0em}
    \setlength{\parskip}{0.5em}

    % Bibliography and citations
    \usepackage[bibencoding=utf8, style=apa]{biblatex}
    \bibliography{../../../../bibliographies/macrofinance, informal}
    
    \title{Central Bank with negative capital} % FIXME: Think of a better title

    \subtitle{A critical essay on the existing literature}

    \author{Andrea Titton}

% Commands

\newcommand{\citein}[1]{\citeauthor{#1} (\citeyear{#1})}
    
\begin{document}

% Title
\pagenumbering{gobble} % stop page numbering
\clearpage
\thispagestyle{empty}
\maketitle
\clearpage

\pagenumbering{arabic}

% --- Sections idea
\iffalse
    History | Theory of CB dividends | Financial stability under deferred assets |
    Expectations of Inflation | Effects on global banks
\fi
% ---

% Main

\section{Introduction}

In developing countries central banks' independence and operability are often taken for granted. Rarely are issues of credibility or financial stability of monetary institutions discussed. Nevertheless, the ``new-style'' central banking (i.e. unconventional monetary policy) calls for a perusal of the banks' balance sheet and the interplay between its financing and its objectives.

The financial crisis of 2008 and the ongoing pandemic economic crisis will leave behind high levels of government debt across the world ($130\%$ world debt-to-GDP ratio according to the \cite{WEO2020}). This places economic and political pressure on central banks to alleviate the debt burden, brought about by necessary fiscal expansions, by means of accommodating monetary policy. Such pressure is commonly referred to as ``fiscal dominance''.

On the one hand, a period of fiscal dominance requires central banks to divert monetary tools towards lowering interest rates on government debt in the secondary market. This might undermine their independence and impede achieving the mandated price stability (\cite{Schnabel2020}). This aspect, albeit fundamental, is not the focus of this analysis. On the other, prologued fiscal dominance might shift financial instability from governments to central banks, deteriorating their capital. %FIXME: Schnabel is improperly cited

This essay analyses, using the relevant literature, the phenomenon and consequences of deteriorating and potentially negative central bank capital. % TODO: Finish this

\section{Defining financial stability and capital}

The financial position, hence the financial stability, of a central bank can be complicated to define. As \citein{Hall2015} point out, relying on conventional definitions applied to commercial banks is not helpful. First of all, the central bank liabilities (i.e. reserves and money issued) cannot be (further) liquidated since, in normal times, they are themselves the fundamental unit of liquidity. Second, the central bank mandate is that of price stability and not profit. This renders the notion of market value for a central bank unapplicable. Third and last, a sizeable and increasing share of the bank's assets is composed by assets of its sole shareholder (i.e. government debt). These structural aspects distort incentives which are present in commercial banking, thereby rendering financial theory around bank solvency and profitability irrelevant.



% Bibliography

\newpage
\nocite{*}
\pagenumbering{gobble} % stop page numbering
\printbibliography

\end{document}
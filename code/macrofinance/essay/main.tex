\documentclass[american]{scrartcl}
    \usepackage{babel}
    \usepackage[utf8]{inputenc} 
    \usepackage{csquotes}
    \usepackage{amsmath}
    \usepackage{graphicx}    

    
    \setlength{\parindent}{0em}
    \setlength{\parskip}{0.5em}

    % Bibliography and citations
    \usepackage[bibencoding=utf8, style=apa]{biblatex}
    \bibliography{../../../../Desktop/bibliographies/macrofinance, informal}
    
    \title{The effects of fiscal dominance \\ on Central Banks' stability}

    % \subtitle{A critical essay on the existing literature}

    \author{Andrea Titton}

% Commands

\newcommand{\citein}[1]{\citeauthor{#1} (\citeyear{#1})}
    
\begin{document}

% Title
\pagenumbering{gobble} % stop page numbering
\clearpage
\thispagestyle{empty}
\maketitle
\clearpage

\pagenumbering{arabic}

% --- Sections idea
\iffalse
    History | Theory of CB dividends | Financial stability under deferred assets |
    Expectations of Inflation | Effects on global banks
\fi
% ---

% Main

\section{Introduction}

In developing countries central banks' independence and operability are often taken for granted. Issues of credibility or financial stability of monetary institutions have taken a backseat in the discussion. Nevertheless, the ``new-style'' of central banking (i.e. unconventional monetary policy) calls for a perusal of the banks' balance sheet and the interplay between its financing and its objectives.

The financial crisis of 2008 and the ongoing pandemic economic crisis will leave behind high levels of government debt across the world (\cite{WEO2020}). This places economic and political pressure on central banks to alleviate the debt burden, brought about by necessary fiscal expansions, by means of accommodating monetary policy. Such pressure is commonly referred to as ``fiscal dominance''.

On the one hand, a period of fiscal dominance requires central banks to divert monetary tools towards lowering interest rates on government debt in the secondary market. This might undermine their independence and impede achieving the mandated price stability (\cite{Schnabel2020}\comment{FIXME: improper citation}). This aspect, albeit fundamental, is not the focus of this analysis. On the other, prologued fiscal dominance might shift financial instability from governments to central banks, deteriorating their capital.

This essay analyses, using the relevant literature, the phenomenon and consequences of deteriorating and potentially negative central bank capital. The focus is on both domestic and international implications of financial instability of the monetary authority. Both issue are discussed from an historical, theoretical, and empirical point of view. First, I give two historical examples of financial instability, the Bank of Amsterdam in the seventeenth century and the cooperation between the Bank of France and England in the eighteenth century. Second, using a modern framework, I define financial stability of a monetary authority and discuss the domestic implications of negative capital. Third, I present a second theoretical framework, embedding the central bank in an international framework, and consider matters of international cooperation under financial instability.

On the one hand, a period of fiscal dominance requires central banks to divert monetary tools towards lowering interest rates on government debt in the secondary market. This might undermine their independence and impede achieving the mandated price stability (\cite{Schnabel2020}). This aspect, albeit fundamental, is not the focus of this analysis. On the other, prologued fiscal dominance might shift financial instability from governments to central banks, deteriorating their capital. % FIXME: Schnabel is improperly cited

This essay analyses, using the relevant literature, the phenomenon and consequences of deteriorating and potentially negative central bank capital. % TODO: Finish this

% TODO: Historical section

\section{Financial stability of central banks}

\subsection{Definition capital and stability}

The financial position, hence the financial stability, of a central bank can be complicated to define. As \citein{Hall2015} point out, relying on conventional definitions applied to commercial banks is not helpful. First of all, the central bank liabilities (i.e. reserves and money issued) cannot be (further) liquidated since, in normal times, they are themselves the fundamental unit of liquidity. Second, the central bank mandate is that of price stability and not profit. This renders the notion of market value for a central bank unapplicable. Third and last, a sizeable and increasing share of the bank's assets is composed by assets of its sole shareholder (i.e. government debt). These structural aspects distort incentives which are present in commercial banking, thereby rendering financial theory around bank solvency and profitability irrelevant.

One needs then to focus on the actual (i.e. those that pose a real resource constraint on its operations) obligations a central bank faces. The bank is \textit{de facto} an independent arm of the government, to which it owes (or is owed) dividends following an implicit dividend rule. The central bank can always meet its obligation towards the government by issuing reserves but in doing so, if the path of reserve issuance is ``explosive'', it loses its independence and ability to fulfill its mandate. % FIXME: Rephrase this

Hence, hereafter central bank insolvency will be understood as the unwillingness of agents to hold its liabilities (i.e. currency) as a consequence of an unstable path of reserve issuance. In a theoretical model this is manifest in hyperinflation, namely $p \xrightarrow{} \infty$, and historically this was followed by a flight to other currencies or the emergence, by political institution or private competition, of a new monetary institution (\cite{Flandreau2007}). % FIXME: Find a better reference

\subsection{Dividend rules}

As we have seen in the previous section, the driving force behind central bank solvency is its dividend rule. Namely, a bank loses independence, hence solvency, if its dividend rule requires a negative and unstable dividends path. Note that, given this definition a central bank can be solvent with negative capital (i.e. negative dividends) if not in an unstable path.

As noted by \citein{Hall2015}, in practice central banks pay dividends either following a \textit{mark-to-market} or a \textit{nominal net worth} rule. Both formulations are, in real terms, equivalent. Dividends are coupon payments from domestic and foreign assets (mostly government bonds), capital gains on such assets, and seigniorage profit net of interest payments on reserves. Such a rule has two major implications. First, negative dividends can arise only as a result of either impaired bond repayments, an appreciation of the real exchange rate, or capital losses. Second, if the dividend rule is constraint to one-period solvency and the central banks only holds domestic risk-free assets, dividends are equivalent to seigniorage revenues, hence cannot be negative.

\subsection{Negative capital}

Most central banks own foreign and/or risky debt, hence the possibility of negative dividends arises from one of the three channels mentioned. Assuming no recapitalization (which is equivalent to insolvency), negative dividends need to be accounted for as deferred assets, namely central banks' future claim on its own future remittances to the Treasury, and their present value needs to be equal to the present value of subsequent positive dividends (\cite{Archer2013}). \citein{Reis2015} proposes to use the size of these deferred asset as a measure of financial stability. The core argument is that, in the limit, the present value of seigniorage sets a threshold over which the central bank's independence is in doubt, since above this level a recapitalization is required at some point in the future.

Given this description it is perfectly possible for a central bank to run sustained negative capital, as long as the deferred assets account is lower than the present value of seigniorage. This result simply states that capital might as well be negative because it is not the core metric of a central bank's financial stability. Furthermore, it requires one to focus on the effects of fiscal dominance on the present value of seigniorage and inflation expectation, to make a claim around solvency in the post-crisis period. % FIXME: What?  

% TODO: Present value of seigniorage is shit to measure and very volatile. Range from 12% to 98% of GDP.

% Bibliography

\newpage
\nocite{*}
\pagenumbering{gobble} % stop page numbering
\printbibliography

\end{document}
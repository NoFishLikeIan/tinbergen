\documentclass[american]{scrartcl}
    \usepackage{babel}
    \usepackage[utf8]{inputenc} 
    \usepackage{csquotes}
    \usepackage{amsmath}
    \usepackage{graphicx}    
    
    \setlength{\parindent}{0em}
    \setlength{\parskip}{0.5em}

    % Bibliography and citations
    \usepackage[bibencoding=utf8, style=apa]{biblatex}
    \bibliography{../../../../Desktop/bibliographies/macrofinance, informal}
    
    \title{The effects of fiscal dominance on the stability and independence Central Banks}

    % \subtitle{A critical essay on the existing literature}

    \author{Andrea Titton}
    

% Commands

\newcommand{\citein}[1]{\citeauthor{#1} (\citeyear{#1})}
\newcommand{\comment}[1]{\iffalse#1\fi}
    
\begin{document}

% Title
\pagenumbering{gobble} % stop page numbering
\clearpage
\thispagestyle{empty}
\maketitle
\clearpage

\pagenumbering{arabic}

% Main

\section{Introduction}

In developed countries central banks' independence and effectiveness are often taken for granted. Issues of credibility or financial stability of monetary institutions have taken a backseat in the discussion. Nevertheless, the ``new-style'' of central banking, which inflated central banks' balance sheets, calls for a review of the interplay between its financing and objectives (\cite{Hall2015}).

The financial crisis of 2008 and the ongoing pandemic economic crisis will leave behind high levels of government debt across the world (\cite{WEO2020}). This places economic and political pressure on central banks to alleviate the debt burden, brought about by necessary fiscal expansions, by means of accommodating monetary policy. Such pressure is commonly referred to as fiscal dominance.

First, a period of fiscal dominance requires central banks to divert monetary tools towards lowering interest rates on government debt in the secondary market. This might undermine their independence and impede achieving the mandated price stability (\cite{FernandezAlbertos2015}). This aspect, albeit fundamental, is not the focus of this analysis. Second, prolonged fiscal dominance might shift financial instability from governments to central banks, deteriorating their capital.

This essay analyses the phenomenon and consequences of deteriorating and potentially negative central bank capital. The focus is on both domestic and international implications of financial instability of the monetary authority. Both issues are discussed from a theoretical and an historical point of view. First, using a modern framework, I define financial stability of a monetary authority and discuss the domestic implications of negative capital. Second, I present an extended theoretical framework, embedding the central bank in an international economy, and consider matters of international cooperation under financial instability. Third, I give an historical account of financial instability, the Bank of Amsterdam in the seventeenth century, and of monetary policy under fiscal dominance, the Banco Central do Brasil in 2002.

% TODO: Maybe also an historical account on cooperation: the Bank of France and England in the eighteenth century

\section{Financial stability of central banks}

\subsection{Central bank's instruments}

A central bank has two main channels through which it interacts with the economy: controlling currency and holding reserves. Furthermore, the bank commits to allow exchange between currency and reserves (informally depositing currency) at a fixed one-to-one rate, such that the two represent the unit of accounting in the economy. From this perspective, the central bank is a government monopolist that produces and sells currency and collects an inflation tax (seigniorage) on currency holders (\cite[p.~15]{Reis2016}).

% FIXME: Clarify

In practice, central banks expand reserves by setting an interest rate, thereby inducing an inflation level, and subsequently accommodate the demand for currency in the economy (\cite[p.~17]{Reis2016}). Framing the operations of a central bank this way implies that the institution can be seen as a bank raising revenue through seigniorage, with the government as sole shareholder, and depositors of reserves as debt holders. Therefore, insofar as reserves are backed by the government, the central bank solvency is equivalent to solvency of the overall government (\cite{Sims2001}).


\subsection{Definition of capital and stability}

The financial position of a central bank can be complicated to define. As \citein{Hall2015} point out, relying on conventional definitions applied to commercial banks is not helpful. First of all, the central bank liabilities (i.e. reserves) cannot be liquidated since, in normal times, they are themselves the fundamental unit of liquidity. Second, the central bank mandate is that of price stability and not profit. This renders the notion of market value for a central bank inapplicable. Third, a sizeable and increasing share of the bank's assets is composed by assets of its sole shareholder (i.e. government debt). These structural aspects distort incentives which are present in commercial banking, thereby rendering financial theory around commercial banks' solvency and profitability irrelevant.

One needs then to focus on the actual obligations a central bank faces (i.e. those that pose a real resource constraint on its operations). In normal times, the bank is \textit{de facto} an independent arm of the government, to which it owes (or is owed) dividends following an implicit dividend rule. The central bank can always meet its obligation towards the government by issuing reserves. Nevertheless, if the path of reserve issuance is ``explosive'', agents would be unwilling to hold reserves within the bank unless the government is expected to recapitalize the bank at some point in the future. Such expectations implicitly put a strain on the independence of the bank from the government and the ability to fulfill its mandate (\cite{deHaan2016}).

% TODO: Add explanation in nevertheless

Hence, central bank insolvency can be understood as the unwillingness of agents to deposit reserves or, equivalently, hold currency, as a consequence of an unstable path of reserve issuance. In a theoretical model this is manifested in hyperinflation and historically this was followed by a flight to other currencies or the emergence of a new monetary institution, either by political institution or private competition (\cite{Flandreau2007}). % FIXME: Find a better reference

\subsection{Dividend rules}

As discussed above, the driving force behind central bank solvency is the relation between its liabilities and its dividends. In this context, negative dividends are a \textit{de facto} form of recapitalization. Recapitalization is necessary when the central bank is incapable of raising revenues via reserves. Namely, a bank loses independence, hence solvency, if its dividend rule requires a negative and unstable dividends path. Note that, given this definition, a central bank can be solvent with negative capital (i.e. negative dividends) if not on an unstable path.

As pointed out by \citein{Hall2015}, in practice central banks pay dividends either following a \textit{mark-to-market} or a \textit{nominal net worth} rule. Both formulations are, in real terms, equivalent: dividends are financed via coupon payments from domestic and foreign assets (mostly government bonds), capital gains on such assets, and seigniorage profit net of interest payments on reserves. Such a rule has two major implications. First, negative dividends can arise only as a result of either impaired bond repayments, an appreciation of the real exchange rate, or capital losses. Second, if the dividend rule is constrained to one-period solvency and the central banks only holds domestic risk-free assets, dividends are equivalent to seigniorage revenues, hence cannot be negative.

\subsection{Negative capital} \label{sec_negative}

Most central banks own foreign and/or risky debt, hence the possibility of negative dividends arises from one of the three channels mentioned above. Outside the explosive reserves path, negative dividends need to be accounted for as deferred assets, namely central banks' future claim on its own future remittances to the Treasury, and their present value needs to be equal to the present value of subsequent positive dividends (\cite{Archer2013}).

\citein{Reis2015} proposes to use the size of these deferred assets as a measure of financial stability. The argument is that, in the limit, the present value of seigniorage, which is the only central bank's form of revenue impossible to lose, sets a threshold to the deferred assets account over which its independence is in doubt, since above this level a recapitalization is required at some point in the future.

Given this description it is perfectly possible for a central bank to run sustained negative capital, as long as the deferred assets account is lower than the present value of seigniorage. This result simply states that capital might as well be negative because it is not the core metric of a central bank's financial stability. Furthermore, it shows that to make a claim around solvency in the post-crisis period, it is necessary to focus on the effects of fiscal dominance on the present value of seigniorage and inflation expectation.

\subsection{Seigniorage}

The capability of a central bank to operate with small or negative capital and to accommodate fiscal policy depends on its capacity to extract seigniorage. As most sources of government revenue, it is a convex problem and a particularly complicated one to put numbers on. Following \citein{Reis2016} and \citein{Hilscher2014}, the central bank can practically extract seigniorage revenue either by printing banknotes and raising inflation or increasing the reserves requirements for banks.

By supplying banknotes beyond their demand, the central bank induces some form of inflation, either in consumer products or assets. As inflation acts as a tax, since it transfers real purchasing power from economic agents to the central bank, one can think of a Laffer curve of banknotes (\cite[p.~17]{Reis2016}). The potential seigniorage via banknote issuance is then the peak of the curve. This has been estimated by \citein{Hilscher2014} to be, in present value terms, below 30\% of GDP in the United States.

The other source of seigniorage revenue, expanding reserves via requirements, poses a more challenging estimation problem. The complication arises because such a policy increases the marginal costs of banks and therefore affects investment levels in the economy. This implies that assessing the present value of revenues generated via reserve requirements requires a general equilibrium model of the economy. Given the uncertainty embedded in these models and the risk of a sudden increase in reserves triggering a financial crisis, this method is very costly (\cite{Hoggarth2002}).

\section{Monetary risks of fiscal dominance}

So far I have focused on the capability of a central bank to alleviate fiscal burdens and the strain this poses to its financial position. In this section I will go more in depth on how a situation of fiscal dominance and a precarious financial position might hinder the central bank's ability to achieve price stability.

\subsection{Inflation targeting}

An important tool central banks have at their disposal to achieve price stability is inflation targeting. All developed countries have a central bank which adopted some form of inflation targeting as a price stability measure in the last 20 years (\cite{Ahmed2021}). As such, it is important to think about the influence of fiscal dominance on inflation targeting. The interplay between the two can be framed, as usual, in a theoretical and an empirical point of view.

A theoretical model of monetary policy in periods of fiscal dominance was laid down by \citein{Woodford1998}. The analysis focuses first on how even a fully independent central bank needs to react to fiscal news. This needs to be true because an economy with non-Ricardian fiscal policy (i.e. a shock in public debt is not one-to-one transferred to the present value of government surplus) can lead to rational expectation equilibria with unstable prices, since the shock affects the households' intertemporal budget constraint, hence their consumption decisions (this is formally shown in \cite[p. 123]{Woodford1998}). Under such a regime, the central bank is forced to intervene in debt management to stabilize price levels since the debt level and composition affect the inflation expectations of agents. \citein{Woodford1998} then concludes that “rather than simply implementing an interest-rate rule and letting the government budget evolve as it may, it would be appropriate for the central bank to play an active role in commenting upon the inflationary consequences of proposed changes in fiscal policy”.

The empirical point of view was taken by \citein{Ahmed2021}. The authors provide an extensive empirical account on the degree to which fiscal dominance, measured as real GDP to debt in local currency ratio, induces debt relief via monetary policy, measured as the deviation from a pre-estimated Taylor rule, controlling for unobserved shocks and other omitted variables. They find that, even among central banks that enjoy a strong independence, there is evidence of accommodating monetary policy and increased debt level arising simultaneously (\cite[p. 19]{Ahmed2021}). Nevertheless, the authors are not able to disentangle the overall trend of lowering interest rates and the variation induced by monetary policy. Hence, the effect of fiscal dominance on inflation targeting is still empirically unclear.

\subsection{Exchange rates and zero lower bound} \label{caballero}

Another fundamental channel that affects price stability and is influenced by both fiscal and monetary policy is the exchange rate. To discuss the matter, I will rely on the theoretical framework of \citein{Caballero2016} and, again, the empirical analysis of \citein{Ahmed2021}.

\citein{Caballero2016} build a model of global imbalances in current account levels in a time when interest rates are close to zero and cannot therefore fall further to clear international asset markets (i.e. zero lower bound). Such a framework is particularly interesting in the context of this essay because it allows one to think of the spillovers that the behavior of central banks and government have on foreign economies.

Assume two equal and financially integrated countries that find themselves in a permanent liquidity trap. This implies that both economies have a positive output gap. In this context, there is a Keynesian output multiplier associated with public debt issuance, due to the positive feedback loop of fiscal inflation, and interest rates are below the autarky level, by definition of liquidity trap (\cite[p. 41]{Caballero2016}). Furthermore, public debt and money are perfect substitute zero interest rate government liabilities, hence they are rigorously equivalent (\cite[p. 42]{Caballero2016}). In such a regime it is very tempting for fiscal authorities to stimulate the economy via public debt but this puts central banks in a difficult position. A fiscal expansion is either absorbed by inflation, if the targeting is weak and prices are flexible, or by an appreciation of the home currency. In the former case, curbing inflation (or not accommodating inflation expectations) reduces the fiscal multiplier by breaking the positive feedback loop. In the latter, there is a strong temptation to depreciate currency via monetary policy. The author's hence showed that ``exchange rate policies generate powerful beggar-thy-neighbor effects on output'' (\cite[p. 32]{Caballero2016}). This model adds international forces, on top of the domestic ones layed out by Woodford, that might force central banks to forget their price stability mandate and incur in financial instability due to fiscal dominance.

Empirically \citein{Caballero2016} point out how, in reaction to the accommodating monetary policy by the FED, the Bank of Japan in 2013 and the ECB in 2014 implemented expansionary monetary policy that strongly depreciated their currencies vis-à-vis the US dollar. Such an observation is nothing more than a stylized fact and does not constitute evidence for the mechanism described in the model. \citein{Ahmed2021} attempt a more subtle empirical analysis, dividing countries into three bins based on the exchange rate regimes they normally employ, and find large heterogeneity in the reaction to the advent of fiscal dominance. Once again, the empirical analysis is inconclusive and requires one to trust the theoretical model.

\section{Examples from history}

In this section the essay will focus on two historical examples of central bank decisions under fiscal dominance. Despite the vastly different context of the two periods, both serve well at illustrating the theory presented so far.

\subsection{Financial instability, the Bank of Amsterdam}

In 1683 the publicly owned Bank of Amsterdam started issuing bank florins, banknotes representing the balance of an account within the bank. These balances were without claim and the bank started issuing them to finance open market operations, seigniorage, and loans. In this sense, one can think of this form of currency as analogous to modern fiat money (\cite[p. 3]{Quinn2014}). This change, which came about as consequence of high deposit rates and not as a monetary experiment (\cite[p. 3]{Quinn2014}), created a more liquid accounting unit than the prior standard (i.e. silver and gold coins) and quickly became the dominant international currency.

The bank was not independent but had a price stability mandate. As \citein{Quinn2014} point out, its balance sheet consisted mostly of loans to the City of Amsterdam and the Dutch East India Company (Amsterdam Municipal Archives, 5077/1314). In particular the former were closer to profit-takings since they would be customarily written off. The role of the bank was then to provide liquidity to the economy and use the profit to, on one hand, smooth cash cycles for the Dutch East India Company, and, on the other hand, to alleviate fiscal burdens for the City. As the ability of the bank to generate revenue increased, so did the loans to City of Amsterdam, which pushed the Bank into a sharp negative capital (\cite[p. 5]{Quinn2014}).

Consistent with the theory layed out in section \ref{sec_negative}, the bank operated a deferred asset balance to finance the write-offs of the loans and was able to do so, deep into negative capital, thanks to the ever increasing seigniorage revenue brought about by the central role the banknotes were taking in the international trade market. The ability to accommodate fiscal policy had as a fuel not the Bank's capital, which was lacking, nor the quality of its liabilities, which were exotic for the time, but the willingness to hold its banknotes brought about by the market demand for the new form of money.

The Bank of Amsterdam had a privileged position in the context of the world economy at the time. First, it had a form of first mover advantage in issuing a new form of money, highly liquid and particularly easy to use in the context of trade. Second, the opacity in its balance sheet implied that agents overestimated the quantity of silver and gold held within the bank (\cite[p. 11]{Quinn2014}).

Nevertheless, the experience of the Bank of Amsterdam, other than bringing about a leap forward in the development of modern monetary policy, is evidence, albeit anecdotal, of how a central bank can operate based on the present value of its seigniorage and the demand for its currency rather than on the financial fundamentals of its balance sheet.

% TODO: Talk about first mover, opacity, and how this means that the central bank can do better or worse

\subsection{Fiscal dominance and monetary policy, Banco Central do Brasil}

A more recent case of a central bank having to reconcile fiscal dominance and price stability is that of Brazil around the turn of the millennium. The issue the Brazilian experience poses represents the other side of the coin: given a preexisting high debt level and high risk aversion, how should a central bank act in case of an increase in inflation?

In the early 2000s, the Brazilian government had very high debt levels. The central bank had strongly committed to inflation targeting but in mid-2002, concerns over the government's fiscal position arose due to the imminent election of Lula. The concerns brought about an increase in interest rate on dollar-denominated debt, which worsened the government's default probability, and a sharp depreciation of the Real, which led in turn to an increase in inflation. Standard open-economy macroeconomics suggests that the central bank ought to increase interest rates to curb inflationary pressure (\cite[p. 3]{Blanchard2004}), yet the Brazilian central bank did not take this approach, arguing that inflationary targeting can have perverse effects under fiscal dominance.

This question is analysed extensively by \citein{Blanchard2004}, who developed a model of public finance and risk aversion of foreign inversion, regulated by capital flows and default risk. The main insight of the model is that, given a high level of debt stock, a high degree of risk aversion of foreign investors, or a high proportion of debt denominated in foreign reserve currency, an increase in interest rate leads to a depreciation of the exchange rate. This depreciation can further exacerbate the price stability problem. Assuming this is the case, fiscal policy is the only channel that can effectively curb inflation.

The example of Brazil might seem at first sight irrelevant for central banks of developed countries. In particular, capital flows are not prone to flight to safety in countries whose currency is perceived as having reserve status. Nevertheless, Blanchard shows that the problem might arise even in case of high debt levels and high risk aversion of agents in the presence of a rival currency. Furthermore, the analysis is consistent with the narrative that exchange rate pressure in case of fiscal dominance can hinder the ability of the central bank to guarantee price stability, as presented in section \ref{caballero} (\cite{Caballero2016}).

\section{Conclusion}

In this essay I described the challenges, domestic and international, a period of fiscal dominance brings about for central banks in developed countries.

First, I walked through some theory on the fiscal position of the central bank and its role in interacting with the real economy and the government finances. The conclusion seems clear, traditional measures of financial stability, like capital ratios and debt sustainability, do not pose a budget constraint for the central bank. This does not mean however that there are no real constraints for central banks. The fundamental source of revenue a central bank has to reconcile with its market operations is the present value of expected seigniorage. As long as economic agents believe this revenue is sufficient to cover its operations, the central bank can in principle accommodate, in real terms, fiscal policy. Central becomes then the quantification of seigniorage capacity of central banks which is deeply linked with inflation expectations. Here I only touched briefly on the quantification issue by presenting current best estimates of around 30\% of GDP.

Second, I looked the risk that fiscal dominance poses to monetary policy and price stability. In periods of fiscal dominance, in particular if accompanied by interest rates at the zero lower bound, monetary policy can be challenging. Domestically, inflation targeting needs to be abandoned in case of a non-Ricardian fiscal policy. Internationally, at the zero lower bound there is strong pressure on the central bank to engage in a beggar-thy-neighbor policy of exchange rate, regardless of inflation levels. The empirical evidence around these two theoretical results is not conclusive, yet there is anecdotal evidence of this happening in the period after the 2008 financial crisis and the 2012 European debt crisis.

Lastly, I presented two historical examples which I think are relevant for the dynamics and political challenges a central bank faces under fiscal dominance. To better understand the origins and structure of central banks, I laid down the relevant aspects of the introduction of \textit{de facto} fiat money by the Bank of Amsterdam in 1683. By no means was the Bank of Amsterdam similar to modern central banks, but this historical event gives context on how in practice a central bank is not concerned with traditional financial measures of stability and how it should rather be seen as a supplier of liquidity and a regulator, even today. A more recent example was that of the Banco do Brazil's monetary policy in 2002. Again, the challenges faced by the Brazilian central bank then are fundamentally different by those faced by governors of the FED or ECB in the wake of the pandemic economic crisis. Yet the anecdote is a cautionary tale on how fiscal dominance can distort traditional monetary policy goals and put a lot of pressure on central banks, both domestically and internationally.

In conclusion, this essay presented a collection of facts and models that can help us envision the role central banks can or will play in the world after the debt surge brought about by the current economic crisis. There is little certainty around this, given the peculiar nature and size of the fiscal expansion governments are undertaking, but there is little doubt central banks will have to play a role. What this world will look like remains unclear, but ``what stands in the way, becomes the way'' (Marcus Aurelius, Meditations, book 5, 20).

\newpage
% Bibliography
% \nocite{*}
\pagenumbering{gobble} % stop page numbering
\printbibliography

\newpage
This page would have been blank if it was not for this text.

\end{document}
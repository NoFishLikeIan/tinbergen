\documentclass[american]{scrartcl}
    \usepackage{babel}
    \usepackage[utf8]{inputenc} 
    \usepackage{csquotes}
    \usepackage{amsmath}
    \usepackage{graphicx}    
    
    \setlength{\parindent}{0em}
    \setlength{\parskip}{0.5em}

    % Bibliography and citations
    \usepackage[bibencoding=utf8, style=apa]{biblatex}
    \bibliography{../../../../Desktop/bibliographies/macrofinance, informal}
    
    \title{The effects of fiscal dominance on Central Banks' stability and independence}

    % \subtitle{A critical essay on the existing literature}

    \author{Andrea Titton}
    

% Commands

\newcommand{\citein}[1]{\citeauthor{#1} (\citeyear{#1})}
\newcommand{\comment}[1]{\iffalse#1\fi}
    
\begin{document}

% Title
\pagenumbering{gobble} % stop page numbering
\clearpage
\thispagestyle{empty}
\maketitle
\clearpage

\pagenumbering{arabic}

% Main

\section{Introduction}

In developed countries central banks' independence and effectiveness are often taken for granted. Issues of credibility or financial stability of monetary institutions have taken a backseat in the discussion. Nevertheless, the ``new-style'' of central banking, which led to an inflation central banks' balance sheets, calls for a review of the interplay between its financing and its objectives (\cite{Hall2015}).

The financial crisis of 2008 and the ongoing pandemic economic crisis will leave behind high levels of government debt across the world (\cite{WEO2020}). This places economic and political pressure on central banks to alleviate the debt burden, brought about by necessary fiscal expansions, by means of accommodating monetary policy. Such pressure is commonly referred to as ``fiscal dominance''.

On the one hand, a period of fiscal dominance requires central banks to divert monetary tools towards lowering interest rates on government debt in the secondary market. This might undermine their independence and impede achieving the mandated price stability (\cite{FernandezAlbertos2015}). This aspect, albeit fundamental, is not the focus of this analysis. On the other hand, prologued fiscal dominance might shift financial instability from governments to central banks, deteriorating their capital.

This essay analyses the phenomenon and consequences of deteriorating and potentially negative central bank capital. The focus is on both domestic and international implications of financial instability of the monetary authority. Both issues are discussed from a theoretical and historical point of view. First, using a modern framework, I define financial stability of a monetary authority and discuss the domestic implications of negative capital. Second, I give a historical example of financial instability, the Bank of Amsterdam in the seventeenth century. Third, I present an extended theoretical framework, embedding the central bank in an international economy, and consider matters of international cooperation under financial instability. % TODO: Maybe also an historical account on cooperation: the Bank of France and England in the eighteenth century

\section{Financial stability of central banks}

\subsection{Central bank's instruments}

A central bank has two main channels through which it interacts with the economy, controlling currency and holding reserves. Furthermore, the bank commits to allow exchange between currency and reserves (informally depositing currency) at a fixed one-to-one rate, such that the two represent the unit of accounting in the economy. From this perspective, the central bank is a government monopolist that produces and sells currency and collects an inflation tax (seigniorage) on currency holders (\cite[p.~15]{Reis2016}).

In practice, central banks expand reserves by setting an interest rate, thereby inducing an inflation level, and after accommodate the demand for currency in the economy (\cite[p.~17]{Reis2016}). Framing the operations of a central bank this way implies that the institution can be seen as a bank raising revenue through seigniorage, with the government as sole shareholder, and depositor of reserves as debt holders. Therefore, insofar as reserves are backed by the government, the central bank solvency is equivalent to solvency of the overall government (\cite{Sims2001})


\subsection{Definition of capital and stability}

The financial position of a central bank can be complicated to define. As \citein{Hall2015} point out, relying on conventional definitions applied to commercial banks is not helpful. First of all, the central bank liabilities (i.e. reserves) cannot be liquidated since, in normal times, they are themselves the fundamental unit of liquidity. Second, the central bank mandate is that of price stability and not profit. This renders the notion of market value for a central bank inapplicable. Third and last, a sizeable and increasing share of the bank's assets is composed by assets of its sole shareholder (i.e. government debt). These structural aspects distort incentives which are present in commercial banking, thereby rendering financial theory around commercial bank's solvency and profitability irrelevant.

One needs then to focus on the actual (i.e. those that pose a real resource constraint on its operations) obligations a central bank faces. In normal times, the bank is \textit{de facto} an independent arm of the government, to which it owes (or is owed) dividends following an implicit dividend rule. The central bank can always meet its obligation towards the government by issuing reserves. Nevertheless, if the path of reserve issuance is ``explosive'', agents would be unwilling to hold reserves within the bank unless the government is expected to recapitalize the bank at some point in the future. Such expectations implicitly puts a strain on the independence of the bank from the government and the ability to fulfill its mandate (\cite{deHaan2016}).

Hence, hereafter central bank insolvency will be understood as the unwillingness of agents to deposit reserves or, equivalently, hold currency, as a consequence of an unstable path of reserve issuance. In a theoretical model this is manifested in hyperinflation and historically this was followed by a flight to other currencies or the emergence of a new monetary institution, either by political institution or private competition (\cite{Flandreau2007}). % FIXME: Find a better reference

\subsection{Dividend rules}

As we have seen in the previous section, the driving force behind central bank solvency is the relation between its liabilities and its dividends. In this context, negative dividends are a \textit{de facto} form of recapitalization. The recapitalization is necessary when the central bank is incapable of raising revenues via reserves. Namely, a bank loses independence, hence solvency, if its dividend rule requires a negative and unstable dividends path. Note that, given this definition, a central bank can be solvent with negative capital (i.e. negative dividends) if not on an unstable path.

As pointed out by \citein{Hall2015}, in practice central banks pay dividends either following a \textit{mark-to-market} or a \textit{nominal net worth} rule. Both formulations are, in real terms, equivalent: dividends are financed via coupon payments from domestic and foreign assets (mostly government bonds), capital gains on such assets, and seigniorage profit net of interest payments on reserves. Such rule has two major implications. First, negative dividends can arise only as a result of either impaired bond repayments, an appreciation of the real exchange rate, or capital losses. Second, if the dividend rule is constraint to one-period solvency and the central banks only holds domestic risk-free assets, dividends are equivalent to seigniorage revenues, hence cannot be negative.

\subsection{Negative capital}

Most central banks own foreign and/or risky debt, hence the possibility of negative dividends arises from one of the three channels mentioned above. Assuming no recapitalization, negative dividends need to be accounted for as deferred assets, namely central banks' future claim on its own future remittances to the Treasury, and their present value needs to be equal to the present value of subsequent positive dividends (\cite{Archer2013}).

\citein{Reis2015} proposes to use the size of these deferred assets as a measure of financial stability. The argument is that, in the limit, the present value of seigniorage, which is the only central bank's form of revenue impossible to lose, sets a threshold to the deferred assets account over which its independence is in doubt, since above this level a recapitalization is required at some point in the future. % Clarify that maximum positive dividends = pv seigniorage

Given this description it is perfectly possible for a central bank to run sustained negative capital, as long as the deferred assets account is lower than the present value of seigniorage. This result simply states that capital might as well be negative because it is not the core metric of a central bank's financial stability. Furthermore, it shows that to make a claim around solvency in the post-crisis period, it is necessary to focus on the effects of fiscal dominance on the present value of seigniorage and inflation expectation.

\subsection{Seigniorage}

The capability of a central bank to operate with small or negative capital and to accommodate fiscal policy depends on the its capacity to extract seigniorage. As most sources of government revenue, it is a convex problem and a particularly complicated one to put numbers on. Following \citein{Reis2016} and \citein{Hilscher2014}, the central bank can practically extract seigniorage revenue either by printing banknotes and raising inflation or increasing the reserves requirements for banks.

By supplying banknotes beyond their demand, the central bank induces some form of inflation, either in consumer products or assets. As inflation acts as a tax, since it transfers real purchasing power from economic agents to the central bank, one can think of a Laffer curve of banknotes (\cite[p.~17]{Reis2016}). The potential seigniorage via banknote issuance is then the peak of the curve. This has been estimated by \citein{Hilscher2014} to be, in present value terms, below 30\% of GDP in the United States.

The other source of seigniorage revenue, expanding reserves via requirements, poses a more challenging estimation problem. The complication arises because such a policy increase the marginal costs of banks and therefore affects investment levels in the economy. This implies that assessing the present value of revenues generated via reserve requirements requires a general equilibrium model of the economy. Given the uncertainty embedded in these models and the risk of a sudden increase in reserves triggering a financial crisis, this method is very costly (\cite{Hoggarth2002}).

\section{Monetary risks of fiscal dominance}

So far I have focused on the capability of a central bank to alleviate fiscal burdens and the strain this poses to its financial position. In this section I will go more in depth on how a situation of fiscal dominance and a precarious financial position might hinder the central bank's ability to achieve price stability.

\subsection{Inflation targeting}

The most important aspect of a central bank's mandate that might be affected by fiscal dominance is inflation targeting. All developed country have a central bank which adopted inflation targeting as a price stability measure in the last 20 years (\cite{Ahmed2021}). The influence of fiscal dominance on inflation targeting can be thought of, as usual, from a theoretical and an empirical point of view.

A theoretical model that looks at how to conduct monetary policy in periods of fiscal dominance was layed down by \citein{Woodford1998}. The analysis focuses first on how even a fully independent central bank needs to react to fiscal news. This needs to be true because an economy with non-Ricardian fiscal policy (i.e. a shock in public debt is not one-to-one transferred to the present value of government surplus) can lead to rational expectation equilibria with unstable prices, since the shock affects the households' intertemporal budget constraint, hence their consumption decisions (this is formally shown in \cite[p. 123]{Woodford1998}). Under such a regime, the central bank is forced to intervene in debt management in order to stabilize price levels since the debt level and composition affect the inflation expectations of agents. \citein{Woodford1998} then concludes that “rather than simply implementing an interest-rate rule and letting the government budget evolve as it may, it would be appropriate for the central bank to play an active role in commenting upon the inflationary consequences of proposed changes in fiscal policy”.

The empirical point of view was taken by \citein{Ahmed2021}. The authors provide an extensive empirical account on the degree to which fiscal dominance, measured as real GDP to debt in local currency ratio, induces debt relief via monetary policy, measured as the deviation from a pre-estimated Taylor rule, controlling for unobserved shocks and other omitted variables. They find that, even among central banks that enjoy a strong independence, there is evidence of accommodating monetary policy and increased debt level arising simultaneously (\cite[p. 19]{Ahmed2021}). Nevertheless, the authors are not able to disentangle the overall trend of lowering interest rates to the variation induced by monetary policy. Hence, the effect of fiscal dominance on inflation targeting is still empirically unclear.

\subsection{Exchange rates and zero lower bound}

Another fundamental channel that affects price stability and is influenced by both fiscal and monetary policy is the exchange rate. In order to discuss the matter, I will rely on the theoretical framework of \citein{Caballero2016} and, again, the empirical analysis of \citein{Ahmed2021}.

\citein{Caballero2016} build a model of global imbalances in current account levels in a time when interest rates are close to zero and cannot therefore fall further to clear international asset markets (i.e. zero lower bound). Such a framework is particularly interesting in the context of this essay because allows one to think of the spillovers that the behavior of central banks and government have on foreign economies.

Let's assume two equal and financially integrated countries that find themselves in a permanent liquidity trap. This implies that both economies have a positive output gap. In this context, there is a Keynesian output multiplier associated with public debt issuance, due to the positive feedback loop of fiscal inflation, and interest rates are below the autarky level, by definition of liquidity trap (\cite[p. 41]{Caballero2016}). Furthermore, public debt and money are perfect substitute zero interest rate government liabilities, hence they are rigorously equivalent (\cite[p. 42]{Caballero2016}). In such a regime it is very tempting for fiscal authorities to stimulate the economy via public debt but this puts central banks in a difficult position. A fiscal expansion is either absorbed by inflation, if the targeting is weak and prices are flexible, or by an appreciation of the home currency. In the former case, curbing inflation (or not accommodating inflation expectations) reduces the fiscal multiplier by breaking the positive feedback loop. In the latter, there is a strong temptation to depreciate currency via monetary policy. The author's hence showed that ``exchange rate policies generate powerful beggar-thy-neighbor effects on output'' (\cite[p. 32]{Caballero2016}). This model adds international forces, on top of the domestic ones layed out by Woodford, that might force central banks to forget their price stability mandate and incur in financial instability due to fiscal dominance.

Empirically \citein{Caballero2016} point out how, in reaction to the accommodating monetary policy by the FED, the Bank of Japan in 2013 and the ECB in 2014 implemented expansionary monetary policy that strongly depreciated such currencies vis-à-vis the US dollar. Such an observation is nothing more than a stylized fact and does not constitute evidence for the mechanism described in the model. \citein{Ahmed2021} attempt a more subtle empirical analysis, dividing countries in three bins based on the exchange rate regimes they normally employ, and find large heterogeneity in the reaction to the advent of fiscal dominance. Once again, the empirical analysis is inconclusive and is left to our conscience to assess whether the theoretical model suffices.

\section{Examples from history}

\subsection{Financial instability, the Bank of Amsterdam}

\subsection{Central bank cooperation, the Bank of England and France}

\subsection{Fiscal dominance and monetary policy, Banco Central do Brazil}

\newpage
% Bibliography
% \nocite{*}
\pagenumbering{gobble} % stop page numbering
\printbibliography

\end{document}
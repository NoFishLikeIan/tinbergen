\documentclass{article}

\usepackage{fancyhdr}
\usepackage{extramarks}
\usepackage{amsmath}
\usepackage{amsthm}
\usepackage{amssymb}
\usepackage{amsfonts}
\usepackage{tikz}
\usepackage[plain]{algorithm}
\usepackage{algpseudocode}

\usetikzlibrary{automata,positioning}

%
% Basic Document Settings
%


\lhead{\hmwkAuthorName}
\chead{\hmwkTitle}
\rhead{\firstxmark}
\lfoot{\lastxmark}
\cfoot{\thepage}

\renewcommand\headrulewidth{0.4pt}
\renewcommand\footrulewidth{0.4pt}

%
% Create Problem Sections
%

\newcommand{\enterProblemHeader}[1]{
    \nobreak\extramarks{}{Problem \arabic{#1} continued on next page\ldots}\nobreak{}
    \nobreak\extramarks{Problem \arabic{#1} (continued)}{Problem \arabic{#1} continued on next page\ldots}\nobreak{}
}

\newcommand{\exitProblemHeader}[1]{
    \nobreak\extramarks{Problem \arabic{#1} (continued)}{Problem \arabic{#1} continued on next page\ldots}\nobreak{}
    \stepcounter{#1}
    \nobreak\extramarks{Problem \arabic{#1}}{}\nobreak{}
}

\setcounter{secnumdepth}{0}
\newcounter{partCounter}
\newcounter{homeworkProblemCounter}
\setcounter{homeworkProblemCounter}{1}
\nobreak\extramarks{Problem \arabic{homeworkProblemCounter}}{}\nobreak{}

%
% Homework Problem Environment
%
% This environment takes an optional argument. When given, it will adjust the
% problem counter. This is useful for when the problems given for your
% assignment aren't sequential. See the last 3 problems of this template for an
% example.
%
\newenvironment{homeworkProblem}[1][-1]{
    \ifnum#1>0
        \setcounter{homeworkProblemCounter}{#1}
    \fi
    \section{Problem \arabic{homeworkProblemCounter}}
    \setcounter{partCounter}{1}
    \enterProblemHeader{homeworkProblemCounter}
}{
    \exitProblemHeader{homeworkProblemCounter}
}

%
% Homework Details
%   - Title
%   - Due date
%   - Class
%   - Section/Time
%   - Instructor
%   - Author
%

\newcommand{\hmwkTitle}{Homework\ \#1}
\newcommand{\hmwkDueDate}{September 13, 2019}
\newcommand{\hmwkClass}{Statistics}
\newcommand{\hmwkAuthorName}{\textbf{Titton Andrea}}

%
% Title Page
%

\title{
    \vspace{2in}
    \textmd{\textbf{\hmwkClass:\ \hmwkTitle}}\\
    \normalsize\vspace{0.1in}\small{Due\ on\ \hmwkDueDate\ at 3:10pm}\\
    \vspace{3in}
}

\author{\hmwkAuthorName}
\date{}

\renewcommand{\part}[1]{\textbf{\large Part \Alph{partCounter}}\stepcounter{partCounter}\\}

%
% Various Helper Commands
%

% Useful for algorithms
\newcommand{\alg}[1]{\textsc{\bfseries \footnotesize #1}}

% For derivatives
\newcommand{\deriv}[1]{\frac{\mathrm{d}}{\mathrm{d}x} (#1)}

% For partial derivatives
\newcommand{\pderiv}[2]{\frac{\partial}{\partial #1} (#2)}

% Integral dx
\newcommand{\dx}{\mathrm{d}x}

% Alias for the Solution section header
\newcommand{\solution}{\textbf{\large Solution}}

% Probability commands: Expectation, Variance, Covariance, Bias
\newcommand{\E}{\mathrm{E}}
\newcommand{\Var}{\mathrm{Var}}
\newcommand{\Cov}{\mathrm{Cov}}
\newcommand{\Bias}{\mathrm{Bias}}

% Spacing
\newcommand{\medvspace}{\vspace{3 mm}}
\newcommand{\smallvspace}{\vspace{5 mm}}
\newcommand{\bigvspace}{\vspace{10 mm}}



\begin{document}

\maketitle

\pagebreak

\begin{homeworkProblem}[32]
    Assume, $X \sim \Gamma(\alpha, \lambda)$. Find $\E(1/X)$.
    \bigvspace
    
    \textbf{Theory}\\
    $X \sim \Gamma \xrightarrow{} \E(X) = \int_{0}^{\infty}\frac{\lambda^{\alpha}}{\Gamma(\alpha)} x^{\alpha - 1} e^{-\lambda*x}$
    
    \bigvspace
    \textbf{Solution}\\
    \begin{equation} \label{eq1}
        \begin{split}
            \E(1/X) & = \int_{0}^{\infty}\frac{\lambda^{\alpha}}{\Gamma(\alpha)} x^{\alpha - 1} \frac{1}{x} e^{-\lambda x} dx\\
            & = \int_{0}^{\infty}\frac{\lambda^{\alpha}}{\Gamma(\alpha)} x^{\alpha - 2} e^{-\lambda x} dx\\
            & = \frac{\lambda^{\alpha}}{\Gamma(\alpha)} \int_{0}^{\infty}x^{\alpha - 2} e^{-\lambda x} dx\\
            \text{Given } \beta = \frac{\Gamma(\alpha - 1)}{\lambda^{\alpha - 1}} \text{ multiply by } \beta \beta^{-1}\\
            \Longrightarrow & = \frac{\lambda^{\alpha}}{\Gamma(\alpha)} \beta \int_{0}^{\infty}\beta^{-1}x^{\alpha - 2} e^{-\lambda x} dx\\
        \end{split}
    \end{equation}
    Note that the third term: $$\int_{0}^{\infty}\frac{\lambda^{\alpha - 1}}{\Gamma(\alpha - 1)}x^{\alpha - 2} e^{-\lambda x} dx\\ = 1$$
    Hence,
    \begin{equation}
        \begin{split}
            \E(1/X) & = \frac{\lambda^{\alpha}}{\Gamma(\alpha)} \frac{\Gamma(\alpha - 1)}{\lambda^{\alpha - 1}}\\
            & = \frac{\lambda^{\alpha}}{\Gamma(\alpha - 1) (\alpha - 1)} \frac{\Gamma(\alpha - 1)}{\lambda^{\alpha - 1}}\\
            & = \lambda / (\alpha - 1)
        \end{split}
    \end{equation}
    
    
Note that $\E(1/X)$ is well-defined only if $\alpha > 1$

\end{homeworkProblem}
\pagebreak

\begin{homeworkProblem}[45]
    \textbf{Part a.}\\
    Show that $\E(z) = \mu$
    
    \begin{equation}
        \begin{split}
            \E(z) & = \E( \alpha x + (1-\alpha) y ) = \alpha \E x + (1-\alpha) \E y\\
            & = \alpha \mu + (1-\alpha) \mu = \mu
        \end{split}        
    \end{equation}
    
\bigvspace
\textbf{Part b.}

Define $f \equiv Var(z)$, then: 

\begin{equation}
    \begin{split}
        f = \Var( \alpha x + (1-\alpha) y ) + 2 \Cov(x, y)\\
    \end{split}        
\end{equation}

By def., $X$ and $Y$ are independent $\xrightarrow{} \Cov(x, y) = 0$
\begin{equation}
    \begin{split}
        f & = \alpha^{2} \Var(x) + (1-\alpha)^{2} \Var(y)\\
        & = \alpha^{2} \sigma_{x}^{2} + (1-\alpha)^{2} \sigma_{y}^{2}
    \end{split}
\end{equation}

$min_{\alpha}(f) \xrightarrow{} (\delta \Var / \delta \alpha = 0 \bigwedge \delta \Var / \delta^{2} \alpha > 0)$:
\begin{equation}
    \begin{split}
        \delta \Var / \delta \alpha & = 0\\
        2 \alpha \sigma_{x}^{2} - 2 (1-\alpha) \sigma_{y}^{2} & = 0 \\
        2 \alpha (\sigma_{y}^{2} + \sigma_{x}^{2}) & = 2 \sigma_{y}^{2} \\
        \alpha^{*} & = \sigma_{y}^{2} / (\sigma_{x}^{2} + \sigma_{y}^{2})
    \end{split}
\end{equation}
and
\begin{equation}
\begin{split}
        \delta \Var /  \delta^{2} \alpha & > 0\\
        2 \sigma_{x}^{2} +  2 \sigma_{y}^{2} & > 0 \xrightarrow{} T
\end{split}
\end{equation}

\textbf{Part c.}    

Assume now that $f \equiv var(\frac{X + Y}{2})$, then:
$$
        f = \frac{1}{4} (\sigma_{x}^{2} + \sigma_{y}^{2})        
$$

Given $f$,
\begin{equation}
    \begin{split}
        f < \sigma_{x}^{2} \Longleftrightarrow \frac{\sigma_{x}^{2}}{\sigma_{y}^{2}} > \frac{1}{3}\\
    \end{split}
\end{equation}
\begin{equation}
    \begin{split}
        f < \sigma_{y}^{2} \Longleftrightarrow \frac{\sigma_{x}^{2}}{\sigma_{y}^{2}} < 3\\
    \end{split}
\end{equation}

That implies that the variance of $z(1/2)$ is smaller that the variances of the two individual variables if their ratio lies between $3$ and $3^{-1}$

\end{homeworkProblem}

\end{document}
\documentclass[american]{scrartcl}
    \usepackage{babel}
    \usepackage[utf8]{inputenc} 
    \usepackage{csquotes}
    \usepackage{amsmath}
    \usepackage{amssymb}
    \usepackage{graphicx}   
    \usepackage{mathtools}
    \usepackage{tikz}
    \usepackage{graphicx}

    
    \setlength{\parindent}{0em}
    \setlength{\parskip}{0.5em}

    
    % Bibliography and citations
    \usepackage[bibencoding=utf8, style=apa]{biblatex}
    \bibliography{ref}

    
    \title{Homework II - Advanced Game Theory }

    \author{Andrea Titton}
    

% Commands
\newcommand{\set}[1]{\left\{#1\right\}}
\newcommand{\Real}{\mathbb{R}}
\newcommand{\abs}[1]{\left\lvert #1 \right\rvert}

% Graphs
\usetikzlibrary{positioning}
\tikzset{main node/.style={circle, draw,minimum size=1cm,inner sep=3pt},}

\begin{document}

% Title

\maketitle

\section*{Exercise 1}

\subsection*{(a)}

The benefit function of an allocation can be written as,

\begin{equation}
    b(x) = b^T x, \text{ where } x = \begin{pmatrix}
        x_1 & x_2 & x_3
    \end{pmatrix}^T \text{ and }b^T = \begin{pmatrix}
        2 & 4 & 7
    \end{pmatrix}
\end{equation}

The water distribution problem, constraint by the water allocation vector $e = \begin{pmatrix} 1 & 0 & 0 \end{pmatrix}^T$ is,

\begin{equation}
    x^* = \arg\max_{x}  b^T x, \ \text{ s.t. } A x \leq e,
\end{equation}

where $A = \begin{pmatrix}
        1 & 0 & 0 \\
        1 & 1 & 0 \\
        1 & 1 & 1
    \end{pmatrix}$.

Given the linearity of the problem it is easy to see that the optimal water allocation problem, restricted to a coalition $S$, yields,

\begin{equation} \label{opt_all_river}
    x^*_i = \begin{cases}
        \sum_{i \in S} e_i & \text{ if } i = \max_{j \in S} j \\
        0                  & \text{ otherwise}
    \end{cases}
\end{equation}

So that all the resources available to a coalition goes to the furthest downstream node. Such that for example, $S = N \implies x = \begin{pmatrix} 0 & 0 & 1 \end{pmatrix}^T$.

Using this optimal allocation we obtain the value function,

\begin{equation}
    v(S) = \begin{cases}
        2 & \text{ if } S = \set{1}    \\
        4 & \text{ if } S = \set{1, 2} \\
        7 & \text{ if } S = N          \\
        0 & \text{ otherwise }
    \end{cases}
\end{equation}

\subsection*{(b)}

First we can compute the marginal vector $m^u(v)$, by looking at the permutation $u = (1, 2, 3)$. Then,

\begin{equation}
    m^u(v) = \begin{pmatrix}
        v(1)                 \\
        v(\set{1, 2}) - v(1) \\
        v(N) - v(\set{1, 2})
    \end{pmatrix} = \begin{pmatrix}
        2 &
        2 &
        3
    \end{pmatrix}^T
\end{equation}

Doing the same for $m^l(v)$, with permutation $l = (3, 2, 1)$, we obtain,

\begin{equation}
    m^l(v) =  \begin{pmatrix}
        7 &
        0 &
        0
    \end{pmatrix}^T
\end{equation}

Then, \begin{equation}
    f^e(v) = \frac{1}{2} \cdot (m^l(v) + m^u(v)) = \begin{pmatrix}
        4.5 & 1 & 1.5
    \end{pmatrix}^T
\end{equation}

Finally to compute the Shapley value of the game we can compute the Harsanyi dividends, which are trivially,

\begin{equation}
    \Delta(S)  = \begin{cases}
        2 & \text{ if } S = \set{1}    \\
        2 & \text{ if } S = \set{1, 2} \\
        3 & \text{ if } S = N          \\
        0 & \text{ otherwise }
    \end{cases}
\end{equation}

which yield a Shapley value of,

\begin{equation}
    f^{Sh}(v) = \begin{pmatrix}
        4 & 2 & 1
    \end{pmatrix}^T.
\end{equation}

\subsection*{(c)}

If $e_3 > 1$, then the optimal allocation remains the one described in (\ref{opt_all_river}) since the constraint changes and $\nabla b$ remains constant.

Given this result, in case of the marginal vector $m^l$ or $m^u$, the final node in the permutation would capture all the extra benefit of increased water supply in $e_3$. So that in $m^u$ the allocation for $1$ and $2$ would not change. Likewise in $m^l$ the allocation for $1$ would be,

\begin{equation}
    m^l_1 = b^T \left(\begin{pmatrix}
            0 \\ 0 \\ e_1 + e_3
        \end{pmatrix} - \begin{pmatrix}
            0 \\ 0 \\ e_3
        \end{pmatrix} \right) = b_3 \cdot e_1
\end{equation}

so that 1's payoff does not change, since he only capture his contribution to the added benefit of the coalition. This implies that allocation of 1 and 2 in $f^e$ does not change as well.

In a similar manner the Shapley value does not yield a higher allocation for 1 and 2, since their dividends Harsanyi dividends remain unchanged.

\subsection*{(d)}


In a line graph $[1, n]$, the hierarchical outcomes associated with roots $1$ and $n$ are,

\begin{equation}
    \begin{split}
        h^n_k &= v[1, k] - v[1, k - 1] = m^u_k \\
        h^1_k &= v[k, n] - v[k+1, n] = m^l_k
    \end{split}
\end{equation}

since $F^1_k = \set{k + 1}$ and $F^n_{k + 1} = \set{k}$. In our case, with $n = 3$, this allows use the values of part (b),

\begin{equation}
    \begin{split}
        h^3 = m^u = \begin{pmatrix}
            2 &
            2 &
            3
        \end{pmatrix}^T \\
        h^1 = m^l = \begin{pmatrix}
            7 &
            0 &
            0
        \end{pmatrix}^T.
    \end{split}
\end{equation}

In order to compute $h^2$, using the directed graph $L^2 = \set{(2, 1), (2, 3)}$, yields,

\begin{equation}
    h^2 = \begin{pmatrix}
        v(1) \\ v(\set{1,2,3}) - v(1) - v(2) \\ v(3)
    \end{pmatrix} = \begin{pmatrix}
        2 \\ 5 \\ 0
    \end{pmatrix}
\end{equation}

\subsection*{(e)}

In order for the outcomes to belong to the $Core(v)$ they need to satisfy, $\sum_{j \in N} h_j^i = v(N) = 1$ and $\sum_{j \in S}  h_j^i \geq v(S)$.

The first condition is satisfied since $v(N) = 7 = \sum_{j \in N} h_j^i \ \forall i$. Let's check the second condition for every $S \subset N$. Note that it is sufficient to only check the connected coalitions with $1$ in them, since $1$ has the only source of water. If this is not the case for a coalition $S$, then $v(S) = 0 \leq \sum_{j} h_j^i$. The remaining coalitions are then $\set{1, 2}$ and $\set{1}$. Then,

\begin{equation}
    \begin{split}
        S = \set{1}&, \ (h^1_1, h^2_1, h^3_1) = (7, 2, 2) \geq v(S) = 2 \\
        S = \set{1, 2}&, \ \sum_{j \in S} (h^1_j, h^3_j, h^3_j) = (7, 7, 4) \geq v(S) = 4 \\
    \end{split}
\end{equation}


Hence the second condition is also satisfied and $h^i \in Core(v) \ \forall i$.

\section*{Exercise 2}

The $Core(v^L)$ of the Myerson restricted game $(N, v^L)$ where $L$ is a line graph, is defined as,

\begin{equation}
    Core(v^L) \coloneqq \set{x \in \Real^N: \ \sum_{i \in S} x_i \geq v^L(S) \land \sum_{i \in N} x_i = v^L(N)}.
\end{equation}

The upper equivalent marginal vector $m^u(v^L)$ is the marginal vector associated with the permutation of $N$, $u = (1, 2, \ldots, n)$. Hence we can write,

\begin{equation} \label{mu_def}
    m^u_i(v^L) = v^L[1, i] - v^L[1, i-1].
\end{equation}

Hereafter we will denote the vector $m^u_i$ and the value function $v$ for simplicity.

Let $S$ be a connected coalition, hence it can be represented as $S=[l, r]$. We can then rewrite,

\begin{equation} \label{conn_res}
    \begin{split}
        \cup_{i \in S} [1, i] &= [1, \max_{k \in S} k] = [1, \min_{j \in S}j] \cup \underbrace{[\min_{j \in S}j, \max_{k \in S}k]}_{S}\\
        \implies & [1, r] = [1, l - 1] \cup S \\
        \implies & v[1, r] - v[1, l - 1]  \geq  v(S) \text{ by superadditivity of } v \\
    \end{split}
\end{equation}

Using now equation (\ref{mu_def}).

\begin{equation} \label{incore}
    \begin{split}
        \sum_{i \in S} m^u_i &= \sum_{i \in S = [l, r]} \left(v[1, i] - v[1, i-1]\right) \\
        &= v[1, r] - v[1, r - 1] + v[1, r -1] - v[1,  r - 2] \ldots + v[1, l] - v[1, l-1] \\
        &= v[1, r] - v[1, l - 1] \geq v(S) \text{ by (\ref{conn_res}).}
    \end{split}
\end{equation}

This result can easily be extended to a not connected coalition, since we can rewrite the coalition as a union of disjoint connected sets, namely, $S = [l, i_1] \cup [i_2, i_3] \cup \dots \cup [i_n, r] = \cup_i I_i$ where $I_i \in \mathcal{I}$.This allow us to derive again equation (\ref{conn_res}) as,

\begin{equation}
    \begin{split}
        [1, r] = [1, l - 1] \cup \underbrace{\left(\bigcup_i I_i\right)}_{S} \cup \left([1, r] \cap S\right)^c \\
        v[1, r] - v[1, l - 1] - v\left([1, r] \cap S\right)^c \geq v(S)
    \end{split}
\end{equation}

Then (\ref{incore}) yields,

\begin{equation} \label{greater}
    \begin{split}
        \sum_{i \in S} m^u_i &= \sum_{i \in \cup_i I_i} \left(v[1, i] - v[1, i-1]\right) \\
        &= v[1, r] - v[1, i_n - 1] + v[1, i_{n-1}] - v[1, i_n - 2] \ldots + v[1, i_1] - v[1, l-1] \\
        &\geq v\left(\bigcup_i I_i\right) = v(S),
    \end{split}
\end{equation}

by noting that, via superadditivity,

\begin{equation}
    v\left([1, r] \cap S\right)^c \geq v[1, i_n - 1] - v[1, i_{n-1}] + v[1, i_n - 2] \ldots - v[1, i_1].
\end{equation}

Now consider the case where $S = N$. Trivially $N \in \mathcal{I}$. Furthermore, using equation (\ref{incore}), and $[l, r] = [0, n]$, we obtain,

\begin{equation} \label{Nincore}
    \begin{split}
        \sum_{i \in N} m^u_i &= \sum_{i \in N} \left(v[1, i] - v[1, i-1]\right) \\
        &= v[1, n] - v(\emptyset) = v(N).
    \end{split}
\end{equation}

The equations \ref{greater} and \ref{Nincore} imply that $m^u$ is in the core of the Myerson restricted game.

\section*{Exercise 3}

\subsection*{(a)}

I believe modelling \textit{Absolute Territorial Sovereignty} via \textit{core-stability} is a sensible choice. In particular core stability implies for an allocation $x$ that $\sum_{i \in S} x_i \geq v(S)$. If $S$ is a single country, this trivially requires that the allocation a country receives is greater than its own individual benefit, namely $x_i \geq v(\set{i})$. In the river game $v(\set{i}) = v(e_i)$: a country's benefit from not participating in a coalition (one can say, its outside option) is using all of its own water, consistent with \textit{ATS}. \textit{Core-stability} extends this concept to any coalition, $\sum_{i \in S} x_i \geq v(S)$, namely an allocation needs to make the coalition better off than using the water of its own members.

\subsection*{(b)}

$\alpha$\textit{-TIBS fairness} is satisfied if the gains of cooperating between to coalitions are divided among all members proportionally to a weight vector $\alpha$. This notion is a correct formalization of the notion of \textit{Territorial Integration of all Basin States}. In particular, the latter requires that the ownership over the watercourse is shared among all basin states independently of the entry of the water course. This implies that an allocation which satisfies $\alpha$\textit{-TIBS fairness} and where the weight vector $\alpha$ is independent of the water course entry vector $e$ respects the \textit{TIBS} principle.



\section*{Exercise 4}

\subsection*{(a)}

First notice that $2 \abs{N} \over 3 = 6$. Hence,

\begin{equation}
    v^L(S) = \begin{cases}
        1  \text{ if $S$ is connected and } \abs{S} \geq 6 \\
        0 \text{ otherwise }
    \end{cases}
\end{equation}


This implies that the in the marginal vector $m^u$ the only marginal contribution is given by the sixth node, which is the first in the $u$ permutation to yield $\abs{\set{1, \dots i}} \geq 6$. Hence the $m^u_i = 1 \text{ if } i = 6 \text{ and } 0$ otherwise. Likewise, $m^l_i$ is 1 if $n - i = 6, i = 4$ and $0$ otherwise since, $\abs{\set{4, \dots n}} = 6$. Then $f^e$ is simply the average of the two vectors, namely,

\begin{equation}
    \begin{split}
        m^u &= \begin{pmatrix} 0 &0 &0 &0 &0 &1 &0 &0 &0 \end{pmatrix} \\
        m^l &= \begin{pmatrix} 0 &0 &0 &1 &0 &0 &0 &0 &0 \end{pmatrix} \\
        f^e &= \begin{pmatrix} 0 &0 &0 &0.5 &0 &0.5 &0 &0 &0 \end{pmatrix}
    \end{split}.
\end{equation}

As before, we need to check that $\sum_{i \in N} m^u = \sum_{i \in N} m^l = \sum_{i \in N} f^e = v(N) $. This is satisfied since all allocations sum to $1$. It is also necessary that,

\begin{equation}
    \sum_{i \in S} x_i \geq v(S), \ x \in \set{m^u, m^l, f^e}.
\end{equation}

First note that for any $S$ such that $V(S) = 0$, the condition is trivially satisfied since $x_i \geq 0 \ \forall i$. Let the coalitions that are connected and have order 6 be $M = \set{[1, 6], [2, 7], [3, 8], [4, 9]}$. All of the coalitions in $M$ have value of 1, and a coalition has value of 1 if and only if it is a superset of a coalition in $M$.

Given that $x_i = 1 \implies i \in \set{4, 6} \implies i \in I \  \forall \ I \in M$, it holds that,

\begin{equation}
    \sum_{j \in S} x_j = 1 \implies v(S) = 1 \implies \sum_{j \in S} x_j \geq v(S).
\end{equation}

Therefore the three allocations $m^u, m^l, f^e$ are in the $Core(v^L)$.

\subsection*{(b)}

First note that, for any $i \in N$, one can always construct a coalition $S$ with $v(S) = 0$, such that, $v(S \cup \set{i}) = 1$. For example, $S = \set{3, 4, 6, 7, 8}$ and $i = 5$. This implies that the Myerson value needs to be positive in all its components, namely,

\begin{equation} \label{positive}
    \mu_i > 0 \ \forall i
\end{equation}

Assume that the Myerson value satisfies the first property of the $Core(v^L)$, namely,

\begin{equation}
    \sum_{i \in N} \mu_i = v(N) = 1.
\end{equation}

Now take $S = [1, 6]$ and $S^c = [7, 9]$. Assume that $\mu$ satisfies the second property. This implies,

\begin{equation}
    \sum_{i \in S}\mu_i \geq V(S) = V(N) = 1.
\end{equation}

But $\sum_{i \in S}\mu_i \geq 1$ and $\sum_{i \in N}\mu_i = 1$, imply that $\sum_{i \in S^c}\mu_i = 0$, which contradicts equation (\ref{positive}). Hence $\mu$ cannot satisfy the second property of the $Core(v, L)$ if it satisfies the first, hence $\mu \notin Core(v, L)$

\end{document}

\documentclass[american]{scrartcl}
    \usepackage{babel}
    \usepackage[utf8]{inputenc} 
    \usepackage{csquotes}
    \usepackage{amsmath}
    \usepackage{amssymb}
    \usepackage{graphicx}   
    \usepackage{mathtools}

    
    \setlength{\parindent}{0em}
    \setlength{\parskip}{0.5em}

    
    \title{Homework I - Advanced Game Theory }

    % \subtitle{A critical essay on the existing literature}

    \author{Andrea Titton}
    

% Commands
\newcommand{\set}[1]{\left\{#1\right\}}
\newcommand{\Real}{\mathbb{R}}
\newcommand{\abs}[1]{\left\lvert #1 \right\rvert}

\begin{document}

% Title

\maketitle

\section*{Exercise 1}

\subsection*{(a)}

We can compute the Harsanyi dividends by using the recursive definition,

\begin{equation}
    \Delta_v(T) = v(T) - \sum_{S \subset T} \Delta_v(S).
\end{equation}

This yields,

\begin{equation}
    \begin{split}
        \Delta_v(\set{1}) &= v(1) = 1 \\
        \Delta_v(\set{2}) &= v(2) = 2 \\
        \Delta_v(\set{3}) &= v(3) = 0 \\
        \Delta_v(\set{1, 2}) &= v(\set{1, 2}) - \Delta_v(\set{1}) - \Delta_v(\set{2}) = 0 \\
        \Delta_v(\set{1, 3}) &= v(\set{1, 3}) - \Delta_v(\set{1}) - \Delta_v(\set{3}) = 2 \\
        \Delta_v(\set{2, 3}) &= v(\set{1, 3}) - \Delta_v(\set{1}) - \Delta_v(\set{3}) = 0 \\
        \Delta_v(\set{1, 2, 3}) &= v(\set{1, 2, 3}) - \Delta_v(\set{1, 2}) - \Delta_v(\set{2, 3}) - \Delta_v(\set{1, 3}) = -1  \\
    \end{split}
\end{equation}

\subsection*{(b)}
Let $N(i) \coloneqq \set{T: T \subseteq N \land \ i \in T}$. Using the definition of Shapley value with Harsanyi dividends, we can compute,

\begin{equation} \label{shapley}
    f_i^S = \sum_{T \in N(i)} \frac{1}{\abs{T}} \cdot \Delta_v(T).
\end{equation}

This yields,

\begin{equation}
    \begin{split}
        f^S_1 &= \frac{1}{3} \cdot\Delta_v(\set{1,2,3}) + \frac{1}{2}\cdot \Delta_v(\set{1,2}) + \frac{1}{2}\cdot \Delta_v(\set{1,3}) + \Delta_v(\set{1}) \\
        &= - \frac{1}{3} + 0 + \frac{2}{2} + 1 =  \frac{5}{3}
    \end{split}
\end{equation}

\begin{equation}
    \begin{split}
        f^S_2 &= \frac{1}{3} \cdot\Delta_v(\set{1,2,3}) + \frac{1}{2}\cdot \Delta_v(\set{1,2}) + \frac{1}{2}\cdot \Delta_v(\set{2,3}) + \Delta_v(\set{2}) \\
        &= - \frac{1}{3} + 0 + 0 + 2 =  \frac{5}{3}
    \end{split}
\end{equation}

\begin{equation}
    \begin{split}
        f^S_3 &= \frac{1}{3} \cdot\Delta_v(\set{1,2,3}) + \frac{2}{3}\cdot \Delta_v(\set{1,2}) + \frac{1}{2}\cdot \Delta_v(\set{1,3}) + \Delta_v(\set{3}) \\
        &= - \frac{1}{3} + 0 + \frac{2}{2} + 0 =  \frac{5}{3} = \frac{2}{3}
    \end{split}
\end{equation}

hence,

\begin{equation*}
    f^S = \begin{pmatrix}
        5/3 & 5/3 & 2/3
    \end{pmatrix}
\end{equation*}

\subsection*{(c)}

In order to verify that the core is empty we can use the definition of the core,

\begin{equation*}
    \begin{split}
        &x_i \in C(N, v) \implies \sum_{i \in S} x_i \geq v(S) \\
        &\sum_{i \in N} x_i = v(N)
    \end{split}
\end{equation*}

Let $x = \begin{pmatrix}
        x_1 & x_2 & x_3
    \end{pmatrix}$ be a candidate allocation. Then the second property requires that,

\begin{equation} \label{sum_req}
    x_1 + x_2 + x_3 = v(N) = 4
\end{equation}

The first property, on the other hand, requires

\begin{equation} \label{gr_req}
    \begin{split}
        x_1 &\geq 1 \\
        x_2 &\geq 2 \\
        x_3 &\geq 0
    \end{split}
    \ \ \
    \begin{split}
        x_1 + x_2 &\geq 3\\
        x_1 + x_3 &\geq 3\\
        x_2 + x_3 &\geq 2
    \end{split}
\end{equation}

Combining (\ref{sum_req}) and (\ref{gr_req}) we know that the core allocation requires

\begin{equation}
    x_3 = 0 \implies x_1 \geq 3 \implies x_2 = 0 \Rightarrow\Leftarrow x_2 \geq 2,
\end{equation}

hence there is no allocation $x$ that satisfies (\ref{sum_req}) and (\ref{gr_req}) which implies that $C(N, v) = \emptyset$.

\subsection*{(d)}

Convexity fails for $S = \set{1, 3}$ and $T = \set{2}$. Since,

\begin{equation}
    \begin{split}
        v\left(S \cup T\right) + v\left(S \cap T\right) &< v(S) + v(T)\\
        v(\set{1, 2, 3}) + v(\emptyset) &< v(\set{1, 3}) + v(2) \\
        4 &< 3 + 2
    \end{split}
\end{equation}

Therefore the game is not convex.

\section*{Exercise 2}

The imputation set is defined as,

\begin{equation}
    I(N, v) = \set{ x \in \Real^N: \ \sum_{i \in N} x_i = v(N) \land x_i \geq v(i) \ \forall i \in N }.
\end{equation}



First, we want to show that Shapley value has the first property of the elements in the imputation set,

\begin{equation}
    \sum_{i \in N} f_i^S = v(N).
\end{equation}


Let $N(i) \coloneqq \set{T: T \subseteq N \land \ i \in T}$ and interpret $S \subset T$ to mean every proper subset, $S \in 2^T \setminus T$.

The sum over elements of the Shapley value is then,

\begin{equation} \label{shap_sum_int}
    \begin{split}
        \sum_{i \in N} f_i^S &= \sum_{i \in N} \sum_{T \in N(i)} \frac{1}{\abs{T}} \cdot \Delta_v(T)\\
        &= \sum_{i \in N} \frac{v(N) - \sum_{S \subset N}\Delta_v(S)}{\abs{N}} + \sum_{i \in N} \sum_{T \in N(i) \setminus N} \frac{1}{\abs{T}} \cdot \Delta_v(T) \\
        &= v(N) - \sum_{S \subset N} \Delta_v(S)+ \sum_{i \in N} \sum_{T \in N(i) \setminus N} \frac{1}{\abs{T}} \cdot \Delta_v(T)
    \end{split}
\end{equation}

In the third term, $\sum_{i \in N} \sum_{T \in N(i) \setminus N} \Delta_v(T) / \abs{T}$, each entry $\Delta_v(T)$ appears whenever $i \in T$. There are $\abs{T}$ such terms, hence we can write,


\begin{equation}
    \begin{split}
        \sum_{i \in N} \sum_{T \in N(i) \setminus N} \frac{1}{\abs{T}} \cdot \Delta_v(T) &= \sum_{T \subset N} \abs{T} \cdot \frac{\Delta_v(T)}{\abs{T}} = \sum_{S \subset N} \Delta_v(S).
    \end{split}
\end{equation}

This implies in (\ref{shap_sum_int}) that, $\sum_{i \in N} f_i^S = v(N)$.



Next we want to show that the Shapley value has the second property,

\begin{equation}
    f_i^S \geq v(i).
\end{equation}

The aim is to rewrite the Shapley value for an individual player as $v(i)$ and a positive term. To do so, note that $\{i\} \in N(i)$. Then the Shapley value for $i$ is,

\begin{equation}
    \begin{split}
        f^S_i &= \sum_{T \in N(i)} \frac{1}{\abs{T}} \cdot \Delta_v(T)\\
        &=  \sum_{T \in N(i)} \frac{1}{\abs{T}} \cdot \left( v(T) - \sum_{S \subset T} \Delta_v(S) \right) \\
        &= v(i) + \sum_{T \in N(i) \setminus \set{i} } \frac{1}{\abs{T}} \cdot \left( v(T) - \sum_{S \subset T} \Delta_v(S) \right).
    \end{split}
\end{equation}

We want now to show that

\begin{equation} \label{diff}
    v(T) - \sum_{S \subset T} \Delta_v(S),
\end{equation}

is positive for all $T \in N(i) \setminus \set{i}$.

Consider a set $D = \set{i, j} \in N(i), \ \abs{D} = 2$, then, by super additivity,

\begin{equation} \label{first_step}
    \begin{split}
        v(D) = v\left( \set{i} \cup \set{j} \right) &\geq v(i) + v(j) \\
        v\left( \set{i} \cup \set{j} \right) - \Delta_v(i) - \Delta_v(j) &\geq 0 \\
        \Delta_v(D) = v(D) - \sum_{S \subset D} \Delta_v(S) &\geq 0
    \end{split}
\end{equation}

Next take the extension set $M = D \cup \set{m} \in N(i)$. By super additivity and using (\ref{first_step}),

\begin{equation} \label{second_step}
    \begin{split}
        v(M) = v(D \cup \set{m}) \geq v(D) + v(m) &= \Delta_v(D) + \Delta_v(i) + \Delta_v(j) + \Delta_v(m) \\
        \implies \Delta_v(M) =  v(M) - \sum_{S \subset M} \Delta_v(S) &\geq 0
    \end{split}
\end{equation}

By induction, we can construct any bigger set $T \in N(i)$ as done in (\ref{second_step}) for (\ref{first_step}). Hence for every $T \in N(i)$, (\ref{diff}) is positive.

If $v(T) -\sum_{S \subset T} \Delta_v(S) \geq 0$ for every $T \in N(i)$, then

\begin{equation}
    f^S_i = v(i) + \sum_{T \in N(i) \setminus \set{i} } \frac{1}{\abs{T}} \cdot \left( v(T) - \sum_{S \subset T} \Delta_v(S) \right) \geq v(i).
\end{equation}

\section*{Exercise 3}

\subsection*{(a)}

The function $v$ is the mapping, $v(\emptyset) = 0$,
$v(\set{1}) = 0$,
$v(\set{2}) = 5$,
$v(\set{3}) = 0$,
$v(\set{1, 2}) = 15$,
$v(\set{1, 3}) = 5$,
$v(\set{2, 3}) = 10$,
$v(\set{1, 2, 3}) = 20$.

\subsection*{(b)}

We first compute the Harsanyi dividends,

\begin{equation}
    \begin{split}
        \Delta_v(\set{1}) &= 0, \\
        \Delta_v(\set{2}) &= 5, \\
        \Delta_v(\set{3}) &= 0, \\
        \Delta_v(\set{1, 2}) &= 10, \\
        \Delta_v(\set{1, 3}) &= 5, \\
        \Delta_v(\set{2, 3}) &= 5, \\
        \Delta_v(\set{1, 2, 3}) &= -5
    \end{split}
\end{equation}

Using (\ref{shapley}) and the same procedure as Exercise 1.b we obtain,

\begin{equation}
    f^S = \begin{pmatrix}
        35/6 & 65/6 & 20/6
    \end{pmatrix}
\end{equation}

\subsection*{(c)}

To find the core of the game we need to find the set such that
\begin{equation}
    C(N, v) =
    \begin{dcases}
        \begin{rcases}
            x \in \Real^3: \\
            x_1 \geq 0, \ x_2 \geq 5, \ x_3 \geq 0\\
            x_1 + x_2 \geq 15, \ x_1 + x_3 \geq 5, \ x_2 + x_3 \geq 10 \\
            x_1 + x_2 + x_3 = 20
        \end{rcases}
    \end{dcases}\\
\end{equation}

The condition $x_1 + x_2 = 20 - x_3$ yields $x_3 \in [0, 5]$. By then combining with the other conditions we obtain $x_1 \in [0, 10]$ and $x_2 \in [5, 15]$. Hence the core is,

\begin{equation}
    C(N, v) =
    \begin{dcases}
        \begin{rcases}
            x \in \Real^3: x_1 + x_2 + x_3 = 20 \\
            x_1 \in [0, 10] \\
            x_2 \in [5, 15] \\
            x_3 \in [0, 5]
        \end{rcases}
    \end{dcases}
\end{equation}

\subsection*{(d)}
We can check every combination of $S, T \in 2^N$, for the condition,

\begin{equation}
    v\left(S \cup T\right) + v\left(S \cap T\right) \geq v(S) + v(T).
\end{equation}

Every combination of $S$ and $T$ yields,

\begin{equation*}
    \begin{split}
        S=\set{1}, T=\emptyset&: \ 0+0 \geq 0+0 \\
        S=\set{2}, T=\emptyset&: \ 5+0 \geq 5+0 \\
        S=\set{3}, T=\emptyset&: \ 0+0 \geq 0+0 \\
        S=\set{1, 2}, T=\emptyset&: \ 15+0 \geq 15+0 \\
        S=\set{1, 3}, T=\emptyset&: \ 5+0 \geq 5+0 \\
        S=\set{2, 3}, T=\emptyset&: \ 10+0 \geq 10+0 \\
        S=\set{1, 2, 3}, T=\emptyset&: \ 20+0 \geq 20+0 \\
        S=\set{2}, T=\set{1}&: \ 15+0 \geq 5+0 \\
        S=\set{3}, T=\set{1}&: \ 5+0 \geq 0+0 \\
        S=\set{1, 2}, T=\set{1}&: \ 15+0 \geq 15+0 \\
        S=\set{1, 3}, T=\set{1}&: \ 5+0 \geq 5+0 \\
        S=\set{2, 3}, T=\set{1}&: \ 20+0 \geq 10+0 \\
        S=\set{1, 2, 3}, T=\set{1}&: \ 20+0 \geq 20+0 \\
        S=\set{3}, T=\set{2}&: \ 10+0 \geq 0+5 \\
        S=\set{1, 2}, T=\set{2}&: \ 15+5 \geq 15+5 \\
        S=\set{1, 3}, T=\set{2}&: \ 20+0 \geq 5+5 \\
        S=\set{2, 3}, T=\set{2}&: \ 10+5 \geq 10+5 \\
        S=\set{1, 2, 3}, T=\set{2}&: \ 20+5 \geq 20+5 \\
        S=\set{1, 2}, T=\set{3}&: \ 20+0 \geq 15+0 \\
        S=\set{1, 3}, T=\set{3}&: \ 5+0 \geq 5+0 \\
        S=\set{2, 3}, T=\set{3}&: \ 10+0 \geq 10+0 \\
        S=\set{1, 2, 3}, T=\set{3}&: \ 20+0 \geq 20+0 \\
        S=\set{1, 3}, T=\set{1, 2}&: \ 20+0 \geq 5+15 \\
        S=\set{2, 3}, T=\set{1, 2}&: \ 20+5 \geq 10+15 \\
        S=\set{1, 2, 3}, T=\set{1, 2}&: \ 20+15 \geq 20+15 \\
        S=\set{2, 3}, T=\set{1, 3}&: \ 20+0 \geq 10+5 \\
        S=\set{1, 2, 3}, T=\set{1, 3}&: \ 20+5 \geq 20+5 \\
        S=\set{1, 2, 3}, T=\set{2, 3}&: \ 20+10 \geq 20+10
    \end{split}
\end{equation*}

Since the condition is true for all sets, the game is convex. Furthermore, convexity implies superadditivity, by taking $S, T$ such that $S \cap T = \emptyset$, hence the game is also superadditive.

\section*{Exercise 4}

In order to show that cooperative games are a generalization of hypergraphs, we will provide a construction that maps any cooperative game to an hypergraph and any hypergraph to some cooperative game.

For the former, take the arbitrary cooperative game $(N, v)$ where $N$ is a finite set and $v: 2^N \mapsto \mathbb{R}$ with $v(\emptyset) = 0$. Now define $X \coloneqq N$ and

\begin{equation}
    E \coloneqq \set{S: S \in 2^N \land v(S) > 0 }.
\end{equation}

Note that $(X, E)$ is an hypergraph since $E \subseteq 2^N \setminus \emptyset$. Hence any hypergraph can be constructed by a cooperative game.

For the latter, take the arbitrary hypergraph $(X, E)$. Let $N \coloneqq X$ and

\begin{equation}
    v: 2^N \mapsto \set{0, 1}, \ \
    v(S) \coloneqq \begin{cases}
        1 \  & \text{ if } S \in E    \\
        0 \  & \text{ if } S \notin E
    \end{cases}. \\
\end{equation}

Note then that $(N, v)$ is a cooperative game with a value function with image restricted to $0$ or $1$. Hence, not any cooperative game can be constructed from an hypergraph.

\end{document}
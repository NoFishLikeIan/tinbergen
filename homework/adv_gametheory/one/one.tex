\documentclass[american]{scrartcl}
    \usepackage{babel}
    \usepackage[utf8]{inputenc} 
    \usepackage{csquotes}
    \usepackage{amsmath}
    \usepackage{amssymb}
    \usepackage{graphicx}   
    
    \setlength{\parindent}{0em}
    \setlength{\parskip}{0.5em}

    
    \title{Homework I - Advanced Game Theory }

    % \subtitle{A critical essay on the existing literature}

    \author{Andrea Titton}
    

% Commands
\newcommand{\set}[1]{\left\{#1\right\}}
\newcommand{\Real}{\mathbb{R}}
\newcommand{\abs}[1]{\left\lvert #1 \right\rvert}

\begin{document}

% Title

\maketitle
\newpage

\section*{Exercise 2}

The imputation set is defined as,

\begin{equation}
    I(N, v) = \set{ x \in \Real^N: \ \sum_{i \in N} x_i = v(N) \land x_i \geq v(i) \ \forall i \in N }.
\end{equation}

First, notice that, starting from the definition with Harsanyi dividends, we can rewrite the Shapley, for a single player as the $v(i)$ and a residual term.

Let $N(i) = \set{T: T \subseteq N \land \ i \in T}$ and note that $\{i\} \subseteq N(i)$. Then we can write the Shapley value for $i$ as,

\begin{equation}
    \begin{split}
        f^S_i &= \sum_{T \in N(i)} \frac{1}{\abs{T}} \cdot \Delta_v(T)\\
        &=  \sum_{T \in N(i)} \frac{1}{\abs{T}} \cdot \left( v(T) - \sum_{S \subset T} \Delta_v(S) \right) \\
        &= v(i) + \sum_{T \in N(i) \setminus \set{i} } \frac{1}{\abs{T}} \cdot \left( v(T) - \sum_{S \subset T} \Delta_v(S) \right)
    \end{split}
\end{equation}

where $S \subset T$ implies every proper subset $S \in 2^T \setminus T$.

For a given $T \in N(i)\setminus\set{i}$, we can expand recursively the expression above in order to rewrite the expression in terms of $v$ applied to proper subsets of $T$,

\begin{equation}
    \begin{split}
        v(T) - \sum_{S \subset T} \Delta_v(S) &= v(T) - \sum_{S \subset T} \left(v(S) - \sum_{G \subset S} \Delta_v(G) \right) \\
        &= v(T) - \sum_{S \subset T} \left( v(S)- \sum_{G \subset S} \left(v(G) - \ldots\right) \right)
    \end{split}
\end{equation}

Given this recursive definition, notice that innermost difference will be constructed with a finite number of singletons sets $P$ of the form,

\begin{equation}
    v\left(\bigcup P\right) - \sum_{P} v(P) \geq 0 \ \text{by superadditivity}.
\end{equation}

% TODO: This argument is not the best

By induction then, any difference is positive, which implies that,

\begin{equation}
    \begin{split}
        &v(T) - \sum_{S \subset T} \Delta_v(S) \geq 0 \ \forall T \\
        &\implies v(i) + \sum_{T \in N(i) \setminus \set{i} } \frac{1}{\abs{T}} \cdot \left( v(T) - \sum_{S \subset T} \Delta_v(S) \right) = f_i^S \geq v(i)
    \end{split}
\end{equation}

\end{document}
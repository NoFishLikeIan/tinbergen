\documentclass[american]{scrartcl}
    \usepackage{babel}
    \usepackage[utf8]{inputenc} 
    \usepackage{csquotes}
    \usepackage{amsmath}
    \usepackage{amssymb}
    \usepackage{graphicx}   
    
    \setlength{\parindent}{0em}
    \setlength{\parskip}{0.5em}

    
    \title{Homework I - Advanced Game Theory }

    % \subtitle{A critical essay on the existing literature}

    \author{Andrea Titton}
    

% Commands
\newcommand{\set}[1]{\left\{#1\right\}}
\newcommand{\Real}{\mathbb{R}}
\newcommand{\abs}[1]{\left\lvert #1 \right\rvert}

\begin{document}

% Title

\maketitle

\section*{Exercise 1}

\subsection*{(a)}

We can compute the Harsanyi dividends recursively as,

\begin{equation}
    \begin{split}
        \Delta_v(\set{1}) &= v(1) = 1 \\
        \Delta_v(\set{2}) &= v(2) = 2 \\
        \Delta_v(\set{3}) &= v(3) = 0 \\
        \Delta_v(\set{1, 2}) &= v(\set{1, 2}) - \Delta_v(\set{1}) - \Delta_v(\set{2}) = 0 \\
        \Delta_v(\set{1, 3}) &= v(\set{1, 3}) - \Delta_v(\set{1}) - \Delta_v(\set{3}) = 3 \\
        \Delta_v(\set{2, 3}) &= v(\set{1, 3}) - \Delta_v(\set{1}) - \Delta_v(\set{3}) = 2 \\
        \Delta_v(\set{1, 2, 3}) &= v(\set{1, 2, 3}) - \Delta_v(\set{1, 2}) - \Delta_v(\set{2, 3}) - \Delta_v(\set{1, 3}) = -1  \\
    \end{split}
\end{equation}

\subsection*{(b)}

Using the definition of Shapley value with Harsanyi dividends, we can compute,

\begin{equation} \label{shapley}
    f_i^S = \sum_{T \in N(i)} \frac{1}{\abs{T}} \cdot \Delta_v(T)
\end{equation}

this yields,

\begin{equation*}
    f^S = \begin{pmatrix}
        5/3 & 5/3 & 2/3
    \end{pmatrix}
\end{equation*}

\subsection*{(c)}

In order to verify that the core is empty we can use the definition of the core,

\begin{equation*}
    \begin{split}
        &x_i \in C(N, v) \implies \sum_{i \in S} x_i \geq v(S) \\
        &\sum_{i \in N} x_i = v(N)
    \end{split}
\end{equation*}

Let $x = \begin{pmatrix}
        x_1 & x_2 & x_3
    \end{pmatrix}$ be a candidate allocation. Then the second property requires that,

\begin{equation} \label{sum_req}
    x_1 + x_2 + x_3 = v(N) = 4
\end{equation}

The first property, on the other hand, requires

\begin{equation} \label{gr_req}
    \begin{split}
        x_1 &\geq 1 \\
        x_2 &\geq 2 \\
        x_3 &\geq 0
    \end{split}
    \ \ \
    \begin{split}
        x_1 + x_2 &\geq 3\\
        x_1 + x_3 &\geq 3\\
        x_2 + x_3 &\geq 2
    \end{split}
\end{equation}

Combining (\ref{sum_req}) and (\ref{gr_req}) we know that the core allocation requires

\begin{equation}
    x_3 = 0 \implies x_1 \geq 3 \implies x_2 = 0 \Rightarrow\Leftarrow x_2 \geq 2,
\end{equation}

hence there is no allocation $x$ that satisfies (\ref{sum_req}) and (\ref{gr_req}) which implies that $C(N, v) = \emptyset$.

\subsection*{(d)}

Convexity fails for $S = \set{1, 3}$ and $T = \set{2}$. Since,

\begin{equation}
    \begin{split}
        v\left(S \cup T\right) + v\left(S \cap T\right) &< v(S) + v(T)\\
        v(\set{1, 2, 3}) + v(\emptyset) &< v(\set{1, 3}) + v(2) \\
        4 &< 3 + 2
    \end{split}
\end{equation}

Therefore the game is not convex.

\section*{Exercise 2}

The imputation set is defined as,

\begin{equation}
    I(N, v) = \set{ x \in \Real^N: \ \sum_{i \in N} x_i = v(N) \land x_i \geq v(i) \ \forall i \in N }.
\end{equation}

First, notice that, starting from the definition with Harsanyi dividends, we can rewrite the Shapley, for a single player as the $v(i)$ and a residual term.

Let $N(i) = \set{T: T \subseteq N \land \ i \in T}$ and note that $\{i\} \subseteq N(i)$. Then we can write the Shapley value for $i$ as,

\begin{equation}
    \begin{split}
        f^S_i &= \sum_{T \in N(i)} \frac{1}{\abs{T}} \cdot \Delta_v(T)\\
        &=  \sum_{T \in N(i)} \frac{1}{\abs{T}} \cdot \left( v(T) - \sum_{S \subset T} \Delta_v(S) \right) \\
        &= v(i) + \sum_{T \in N(i) \setminus \set{i} } \frac{1}{\abs{T}} \cdot \left( v(T) - \sum_{S \subset T} \Delta_v(S) \right)
    \end{split}
\end{equation}

where $S \subset T$ implies every proper subset $S \in 2^T \setminus T$.

Consider now, for a given $T$, the term,

\begin{equation} \label{diff}
    v(T) - \sum_{S \subset T} \Delta_v(S).
\end{equation}

In this case $\abs{T} \neq 1$, since the only singleton in $N(i)$ is $\set{i}$ by construction. Now consider a set $T = \set{i, j} \in N(i), \ \abs{T} = 2$, then, by super additivity,

\begin{equation}
    v(T) = v\left( \set{i} \cup \set{j} \right) \geq v(i) + v(j) \implies v(T) - \sum_{S \subset T} \Delta_v(S) \geq 0
\end{equation}

By induction we can construct any bigger set $T$ as union of smaller sets and show that (\ref{diff}) is positive. If $v(T) -\sum_{S \subset T} \Delta_v(S) \geq 0 \ \forall T \in N(i)$, then

\begin{equation}
    f^S_i = v(i) + \sum_{T \in N(i) \setminus \set{i} } \frac{1}{\abs{T}} \cdot \left( v(T) - \sum_{S \subset T} \Delta_v(S) \right) \geq v(i).
\end{equation}

\section*{Exercise 3}

\subsection*{(a)}

The function $v$ is the mapping, $v(\emptyset) = 0$,
$v(\set{1}) = 0$,
$v(\set{2}) = 5$,
$v(\set{3}) = 0$,
$v(\set{1, 2}) = 15$,
$v(\set{1, 3}) = 5$,
$v(\set{2, 3}) = 10$,
$v(\set{1, 2, 3}) = 20$.

\subsection*{(b)}

Using (\ref{shapley}) we obtain,

\begin{equation}
    f^S = \begin{pmatrix}
        35/6 & 65/6 & 10/3
    \end{pmatrix}
\end{equation}

\subsection*{(d)}
We can check every combination of $S, T \in 2^N$, for the condition,

\begin{equation}
    v\left(S \cup T\right) + v\left(S \cap T\right) \geq v(S) + v(T).
\end{equation}

This is true for all sets hence the $G$ is convex. Furthermore, convexity implies superadditivity, by taking $S, T$ such that $S \cap T = \emptyset$, hence the game is also superadditive.

\end{document}
\documentclass[american]{scrartcl}
    \usepackage{babel}
    \usepackage[utf8]{inputenc} 
    \usepackage{csquotes}
    \usepackage{amsmath}
    \usepackage{amssymb}
    \usepackage{graphicx}   
    \usepackage{mathtools}

    
    \setlength{\parindent}{0em}
    \setlength{\parskip}{0.5em}

    
    \title{Homework I - Advanced Game Theory }

    % \subtitle{A critical essay on the existing literature}

    \author{Andrea Titton}
    

% Commands
\newcommand{\set}[1]{\left\{#1\right\}}
\newcommand{\Real}{\mathbb{R}}
\newcommand{\abs}[1]{\left\lvert #1 \right\rvert}

\begin{document}

% Title

\maketitle

\section*{(1) Directed Search}

\subsection*{(a)}

Given pairwise matching, within a submarket the number of firms matching with a worker needs to be equal to the number of workers matching with a firm. Using this we can rewrite,

\begin{equation}
    \begin{split}
        \alpha_w \cdot n_w &= \alpha_v \cdot n_v \\
        \alpha_w &= \alpha_v \cdot \frac{n_v}{n_w} \\
        &= \alpha(n) \cdot \frac{1}{n}
    \end{split}
\end{equation}

\subsection*{(b)}

First notice that the expected value that a worker receives by matching is the extra value, on top of their outside option, they obtain by accepting a job, weighted by the matching probability, namely

\begin{equation}
    V_w = \alpha_w \cdot (w - b) = \frac{\alpha(n)}{n} \cdot (w - b).
\end{equation}

When offering a wage, $w$, thereby attracting a certain amount of workers that generate a queue, $n$, the firm needs to take the workers expected value from matching in account as a participation constraint. On the other hand, the firm picks a wage and queue that maximizes its own expected value from matching, namely,

\begin{equation} \label{firm_opt}
    V_v = \max_{n, w} \left\{ \underbrace{\alpha(n)}_{\text{match probability}} \cdot \underbrace{(y- w)}_{\text{surplus of match}} \right\}
\end{equation}

\subsection*{(c)}

Using $V_w$ we can rewrite the wage as,

\begin{equation}
    w = \frac{n \cdot V_w}{\alpha(n)} + b.
\end{equation}

By plugging in the wage in Equation (\ref{firm_opt}), te optimization problem of the firm reduces to,

\begin{equation}
    V_v = \max_{n} \{ \alpha(n) \cdot (y - b) - n \cdot V_w \}
\end{equation}

Taking the first order condition with respect to $n$ to solve the maximization problem yields,

\begin{equation} \label{firm_opt_condition}
    \alpha^\prime(n) \cdot(y-b) - V_w = 0
\end{equation}

\subsection*{(d)}

Using (\ref{firm_opt_condition}) in the wage definition we can rewrite,

\begin{equation}
    \begin{split}
        w &= \frac{n \cdot \alpha^\prime(n) \cdot(y-b)}{\alpha_n} + b \\
    \end{split}
\end{equation}

Now let $\varepsilon \coloneqq \frac{n \cdot \alpha\prime(n)}{\alpha(n)}$ be the elasticity of $\alpha(n)$ with respect to tightness. Then the wage can be written as,

\begin{equation}
    \begin{split}
        w &= \varepsilon \cdot (y - b) + b \\
        &= \varepsilon \cdot y - (1 - \varepsilon) \cdot b
    \end{split}
\end{equation}

\section*{(2) Albrecht Gautier and Vroman (2016 RED)}



\end{document}
\documentclass[american]{scrartcl}
    \usepackage{babel}
    \usepackage[utf8]{inputenc} 
    \usepackage{csquotes}
    \usepackage{amsmath}
    \usepackage{amssymb}
    \usepackage{graphicx}   
    \usepackage{mathtools}

    
    \setlength{\parindent}{0em}
    \setlength{\parskip}{0.5em}

    
    \title{Homework I - Advanced Game Theory }

    % \subtitle{A critical essay on the existing literature}

    \author{Andrea Titton}
    

% Commands
\newcommand{\set}[1]{\left\{#1\right\}}
\newcommand{\Real}{\mathbb{R}}
\newcommand{\abs}[1]{\left\lvert #1 \right\rvert}

\begin{document}

% Title

\maketitle

\section*{(1) Directed Search}

\subsection*{(a)}

Given pairwise matching, within a submarket the number of firms matching with a worker needs to be equal to the number of workers matching with a firm. Using this we can rewrite,

\begin{equation}
    \begin{split}
        \alpha_w \cdot n_w &= \alpha_v \cdot n_v \\
        \alpha_w &= \alpha_v \cdot \frac{n_v}{n_w} \\
        &= \alpha(n) \cdot \frac{1}{n}
    \end{split}
\end{equation}

\subsection*{(b)}

First notice that the expected value that a worker receives by matching is the extra value, on top of their outside option, they obtain by accepting a job, weighted by the matching probability, namely

\begin{equation}
    V_w = \alpha_w \cdot (w - b) = \frac{\alpha(n)}{n} \cdot (w - b).
\end{equation}

When offering a wage, $w$, thereby attracting a certain amount of workers that generate a queue, $n$, the firm needs to take the workers expected value from matching in account as a participation constraint. On the other hand, the firm picks a wage and queue that maximizes its own expected value from matching, namely,

\begin{equation} \label{firm_opt}
    V_v = \max_{n, w} \left\{ \underbrace{\alpha(n)}_{\text{match probability}} \cdot \underbrace{(y- w)}_{\text{surplus of match}} \right\}
\end{equation}

\subsection*{(c)}

Using $V_w$ we can rewrite the wage as,

\begin{equation}
    w = \frac{n \cdot V_w}{\alpha(n)} + b.
\end{equation}

By plugging in the wage in Equation (\ref{firm_opt}), te optimization problem of the firm reduces to,

\begin{equation}
    V_v = \max_{n} \{ \alpha(n) \cdot (y - b) - n \cdot V_w \}.
\end{equation}

Taking the first order condition with respect to $n$ to solve the maximization problem yields,

\begin{equation} \label{firm_opt_condition}
    \alpha^\prime(n) \cdot(y-b) - V_w = 0.
\end{equation}

Hence, given $\alpha(\cdot)$, $n$ can be retrieved as,

\begin{equation}
    n = (\alpha^\prime)^{-1}\left( \frac{V_w}{y - b} \right)
\end{equation}


\subsection*{(d)}

Using (\ref{firm_opt_condition}) in the wage definition we can rewrite,

\begin{equation} \label{eq_wages}
    \begin{split}
        w &= \frac{n \cdot \alpha^\prime(n) \cdot(y-b)}{\alpha_n} + b. \\
    \end{split}
\end{equation}

Now let $\varepsilon \coloneqq n \cdot \alpha\prime(n) / \alpha(n)$ be the elasticity of $\alpha(n)$ with respect to tightness. Then the wage can be written as,

\begin{equation} \label{eq_wages_el}
    \begin{split}
        w &= \varepsilon \cdot (y - b) + b \\
        &= \varepsilon \cdot y + (1 - \varepsilon) \cdot b.
    \end{split}
\end{equation}

This result simply states that the worker gets a fraction $\varepsilon$ of the surplus generated by the match. Intuitively, imagine an empty queue, such that $n \xrightarrow{} 0$. By de l'Hôpital,

\begin{equation}
    \lim_{n \xrightarrow{} 0} \varepsilon = \lim_{n \xrightarrow{} 0} \frac{n \cdot \alpha^{\prime\prime}(n) + \alpha^\prime(n)}{\alpha^\prime(n)} = 1.
\end{equation}

Hence the first worker to join queue is the one that fully enables the match hence would internalize the full surplus as wage $w = y$. On the other hand, if the queue is already infinitely long, $\varepsilon \xrightarrow{} 0$ as $n \xrightarrow{} \infty$, the marginal effect of a new worker on the matching probability is null, hence such worker would simply be compensated, in case of a match, with his outside option $b$.

\subsection*{(e)}

The Hosios condition holds whenever the externalities generated by the search process are internalized by agents.

In this particular case, as we have seen in the previous question, workers by directing their search and firms by attracting workers, change the matching frictions by increasing the queue. The externality represented by such frictions are internalized by the two sides since the marginal matching probability is immediately reduces as the queue gets longer.

\subsection*{(f)}

Assume now there is an entry fee $k$ for each firm. Firms will enter as long as at the value from entering is equal to the fee, namely,

\begin{equation}
    V_v = \alpha(n) \cdot (y - w) = k.
\end{equation}

Notice that the optimization problem does not change at the margin hence we can still use the wage equilibrium (\ref{eq_wages_el}),

\begin{equation}
    \begin{split}
        k &= \alpha(n) \cdot(y - \varepsilon \cdot y + (1 - \varepsilon) \cdot b)\\
        &= \alpha(n) \cdot (1 - \varepsilon) \cdot (y - b).
    \end{split}
\end{equation}

This equation then uniquely determines the equilibrium.

In order to determine whether this condition is constrained efficient we can solve the planner problem. The planner wants to maximize the expected surplus generated from matching, net of fee payments. This yields the optimization problem,

\begin{equation}
    \max_n \left\{ \underbrace{\frac{\alpha(n)}{n} \cdot (y-b)}_{\text{expected surplus of match}} - \underbrace{\frac{k}{n}}_{\text{fee payments}} \right\}
\end{equation}

Taking the first order condition yields,

\begin{equation}
    (y - b) \cdot \frac{\alpha^\prime(n) \cdot n - \alpha(n)}{n^2} + \frac{k}{n^2} = 0,
\end{equation}

which implies that,

\begin{equation}
    \begin{split}
        k &= \left( \alpha(n) - \alpha^\prime(n) \cdot n \right) \cdot ( y - b) \\
        &= \alpha(n) \cdot \left( 1 - \frac{\alpha^\prime(n) \cdot n}{\alpha(n)} \right) \cdot ( y - b) \\
        &= \alpha(n) \cdot (1 - \varepsilon) \cdot (y - b).
    \end{split}
\end{equation}

Hence the planner problem yields the same condition as the market which implies constrained efficiency.

\section*{(2) Albrecht Gautier and Vroman (2016 RED)}

\subsection*{(a)}

In this model, once a buyer approaches a seller, the game boils down to a \textit{de facto} auction. The buyers will in equilibrium bid their true value (either in the accepting $a$, in the counteroffer phase, or in the English auction phase). Auction theory then implies that there is revenue equivalence among all posting prices $a \in [s, 1]$. Intuitively, the tradeoff between attracting more people and risking too many low valuation buyers is perfectly balanced within this interval. Hence all of these posting prices must allow for an equilibrium.

On the other hand, any posting price $a < s$, increases the probability of a sale, but such potential sales are below the reservation value $s$. Intuitively, attracting more people with valuation below reservation at best yields the same sale value as not doing so. Hence, it must be strictly worse than posting the reservation value $a = s$.

\subsection*{(b)}

Here I will list all possible equilibria. Some of these do not satisfy incentive constraint or might not exist in this game.

Under pure strategy there are three equilibria. First separating, where the asking price is the same as ones reservation value. Then pooling, where all asking prices are equal across sellers, regardless of the reservation value. These can either exhibit an asking price equal to the low reservation value, $a = 0$, or an asking price equal to the high reservation, $a = s$.

Furthermore, there are two mixing equilibria. Either low reservation types mix between high and low asking prices and high types post the true reservation value. Or, vice-versa, the high types mix and the low types reveal their type.

\subsection*{(c)}

The correct equilibrium conditions in a signaling game are those prescribed by the Perfect Bayesian Equilibrium. This definition requires players to play consistent with their best response function conditional on their information set (i.e. both type of buyers and the sellers should not have profitable deviation given the other players best strategies) and it requires for beliefs to be consistent (i.e. the buyer's beliefs over types should be consistent with Bayes rules given the observed posting price of the seller).

This extensions is necessary since the Subgame Perfect Nash Equilibrium concept does not allow for the seller to use the information of the buyer's posting and forces him to behave only based on prior expectations. This confines the game to pooling equilibria only, which we don't always observe in real life.

\end{document}
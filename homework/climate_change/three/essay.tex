\documentclass[american]{scrartcl}
    \usepackage{babel}
    \usepackage[utf8]{inputenc} 
    \usepackage{csquotes}
    \usepackage{amsmath}
    \usepackage{graphicx}    
    
    \setlength{\parindent}{0em}
    \setlength{\parskip}{0.5em}

    % Bibliography and citations
    \usepackage[bibencoding=utf8, style=apa]{biblatex}
    \bibliography{refs}

    
    \title{Subglobal carbon policy \\ and the competitive selection of heterogeneous firms, \\ Balistreri \& Rutherford}
    \subtitle{Critical Essay - Climate Change}

    \author{Andrea Titton}
    

% Commands

\newcommand{\citein}[1]{\citeauthor{#1} (\citeyear{#1})}
\newcommand{\comment}[1]{\iffalse#1\fi}

\iffalse
    What (the research question is)? 

    Does using Melitz for trade changes implications of climate policy?

    Why (is it an important question)? 

    How (do we get to the answer)?

    Results:

    - In non-abating countries energy-intensive industries move to export (competitive advantage)
    -
\fi
    
\begin{document}

% Title
\maketitle

% Main

\section{Summary}

Due to carbon leakage, understanding the effects of any carbon policy requires a reliable theory of international trade. Climate-policy modelling has so far relied on a model developed, between \citeyear{Armington1969} and 1979, by \citeauthor{Armington1969} which assumes perfect competition and country specialization, with the aim of giving a microeconomic foundation to the gravity model of trade. This set of assumptions discounts the firms' heterogeneity in productivity, market power, and decision to export or not. These three aspects of production and trade have significant implications for carbon leakage since climate policy affects the relative costs of energy intensive industries between countries, thereby inducing a productivity and size selection process that influences emission levels.

To overcome these limitations, \citein{Balistreri2012} propose to integrate into an empirical general-equilibrium simulation model the theory of trade developed by \citein{Melitz2003}. \citeauthor{Melitz2003}'s theory of trade relies on monopolistic competition between heterogenous firms that can further decide whether to participate in the export market. This framework allows to determine the evolution of productivity and size of exporting energy firms as a result of a change in relative energy costs induced by carbon policy.

Given the aim of providing a clear comparison between a traditional \citeauthor{Armington1969} approach and the introduction of the \citeauthor{Melitz2003} theory, the authors calibrate all common aspects between the two formulations following the standard \citeauthor{Armington1969} calibration. Having assumed heterogeneity in the energy sector, the authors need to further calibrate, for this industry, the variety substitution coefficient and the productivity distributional parameters. The no policy scenario is used as a baseline case. Then the model is evaluated under two policy scenarios: a reduction of 20\% of the coalition's emissions, and a full border carbon adjustment scenario,  with embodied-carbon tariffs and export rebates on energy intensive trade.

The model simulation highlights a fundamental channel of carbon leakage which is missed in the \citeauthor{Armington1969} formulation. Under the reduction of emission within the coalition scenario, which can be seen as an increased cost for the energy sector in coalition countries, the energy sector in non-coalition countries is given a relative cost advantage. This advantage yields an expansion in the non-coalition in the energy sector, which in turn boosts output and emissions. The total output increase is more than double with respect to the standard \citeauthor{Armington1969} formulation. This result highlights the role of market structure in exacerbating or mitigating carbon leakage. A second important result arises when adding border carbon adjustments. The relative cost advantages arising from emission reductions are offset by increased costs of exporting energy intensive goods. Hence the productivity gains are hindered particularly in countries with large energy intensive production sectors. This significantly reduced leakage and yields results similar to the \citeauthor{Armington1969} formulation.

These results have strong policy implications. First and foremost, they confirm that the cost of joining a coalition, namely losing the relative cost advantage in one's own energy industry, are increasing in the number of members. This hinders the feasibility of gradually increasing coalitions. Second, it shows that productivity effects of climate policy can significantly exacerbate carbon leakage. Lastly, border carbon adjustments are an instrument that can mitigate these additional effects in the policymaker tool belt.

\newpage
\section{Strengths}



\newpage
\section{Weaknesses}

\newpage
\section{Conclusion}


\newpage
% Bibliography
\nocite{*}
\pagenumbering{gobble} % stop page numbering
\printbibliography

\end{document}
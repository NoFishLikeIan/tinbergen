\documentclass[american]{scrartcl}
    \usepackage{babel}
    \usepackage[utf8]{inputenc} 
    \usepackage{csquotes}
    \usepackage{amsmath}
    \usepackage{graphicx}    
    
 

    % Bibliography and citations
    \usepackage[bibencoding=utf8, style=apa]{biblatex}
    \bibliography{refs}

    
    \title{Subglobal carbon policy \\ and the competitive selection of heterogeneous firms, \\ Balistreri \& Rutherford}
    \subtitle{Critical Essay - Climate Change}

    \author{Andrea Titton}
    

% Commands

\newcommand{\citein}[1]{\citeauthor{#1} (\citeyear{#1})}
\newcommand{\comment}[1]{\iffalse#1\fi}

\iffalse
    What (the research question is)? 

    Does using Melitz for trade changes implications of climate policy?

    Why (is it an important question)? 

    How (do we get to the answer)?

    Results:

    - In non-abating countries energy-intensive industries move to export (competitive advantage)
    -
\fi
\linespread{1.25}
\begin{document}

% Title
\maketitle

% Main
\section{Summary}

Due to carbon leakage, understanding the effects of any carbon policy requires a reliable theory of international trade. Climate-policy modelling has so far relied on a model developed, between \citeyear{Armington1969} and 1979, by \citeauthor{Armington1969} which assumes perfect competition and country specialization, with the aim of giving a microeconomic foundation to the gravity model of trade. This set of assumptions does not account for firms' heterogeneity in productivity, market power, and decision to export or not. These three aspects of production and trade have significant implications for carbon leakage since climate policy affects the relative costs of energy intensive industries between countries, thereby inducing a productivity and size selection process that influences emission levels.

To overcome these limitations, \citein{Balistreri2012} propose to integrate into an empirical general-equilibrium simulation model the theory of trade developed by \citein{Melitz2003}. \citeauthor{Melitz2003}'s theory of trade relies on monopolistic competition between heterogenous firms that can further decide whether to participate in the export market. This framework allows to determine the evolution of productivity and size of exporting energy firms as a result of a change in relative energy costs induced by climate policy.

Given the aim of providing a clear comparison between a traditional \citeauthor{Armington1969} approach and the introduction of the \citeauthor{Melitz2003} theory, the authors calibrate all common aspects between the two formulations following the standard \citeauthor{Armington1969} calibration. Having assumed heterogeneity in the energy sector, the authors need to further calibrate, for this industry, the substitution coefficient of product variety, and the productivity distributional parameters. The no policy scenario is used as a baseline case. Then the model is evaluated under two policy scenarios: a reduction of 20\% of the coalition's emissions, and a full border carbon adjustment scenario,  with embodied-carbon tariffs and export rebates on energy intensive trade.

The model simulation highlights a fundamental channel of carbon leakage which is missed in the \citeauthor{Armington1969} formulation. Under the reduction of emission within the coalition scenario, which can be seen as an increased cost for the energy sector in coalition countries, the energy sector in non-coalition countries is given a relative cost advantage. This advantage yields an expansion of the energy-intensive manufacturing sector in non-coalition countries, which in turn boosts output and emissions. The total output increase is more than double with respect to the standard \citeauthor{Armington1969} formulation. This result highlights the role of market structure in exacerbating or mitigating carbon leakage. A second important result arises when adding border carbon adjustments. The relative cost advantages arising from emission reductions are offset by increased costs of exporting energy intensive goods. Hence the productivity gains are hindered particularly in countries with large energy intensive production sectors. This significantly reduced leakage and yields results similar to the \citeauthor{Armington1969} formulation.

These results have strong policy implications. First and foremost, they confirm that the cost of joining a coalition, namely losing the relative cost advantage in one's own energy industry, is increasing in the number of members. This hinders the feasibility of gradually increasing coalitions. Second, it shows that productivity effects of climate policy can significantly exacerbate carbon leakage. Lastly, border carbon adjustments are an instrument, in the policy maker's tool belt, that can mitigate these additional effects.

\section{Strengths}

\citein{Balistreri2012} integrate the theoretical developments by \citein{Melitz2003} on international trade into a climate-policy model. This represents a significant contribution to the climate economics literature since, on one hand, the model by \citeauthor{Melitz2003} addresses explicitly the firm dynamics (productivity and costs) involved in export decision which are a key channel of carbon leakage and, on the other hand, the model has strong empirical backing in the international trade industry (\cite{Helpman2008}), as opposed to the \citein{Armington1969} which is commonly used in climate-policy modelling.

The authors choose to carry out this comparison by applying \citeauthor{Melitz2003}'s insights only to the most relevant sector for carbon leakage (i.e. the energy intensive manufacturing) in order to minimize the difference in calibration with the benchmark model. This allows them to precisely quantify the increased leakage due to carbon emission restrictions and the mitigating effect of carbon-border adjustments. This choice yields a clear comparison of the two frameworks and provides compelling reasons for the climate-policy modelling to adopt the \citeauthor{Melitz2003}'s framework, as opposed to that of \citeauthor{Armington1969}.


\section{Weaknesses}

From a modelling point of view, the choice of restricting \citeauthor{Melitz2003}'s formulation to the energy-intensive manufacturing good, albeit proving useful in the comparison procedure, might be too restrictive. In particular, costs shocks tend to have inter-industry spillovers which are not modeled in the proposed framework. These spillovers can lead to shrinkage of other industries and reallocation of factors of productions, as shown empirically by \citein{Melitz2012}. Furthermore, the Stolper-Samuelson theorem (\cite{Stolper1941}) might apply, which would render a general equilibrium analysis necessary to disentangle the inter-industry consequences of climate policy. From an empirical point of view, the analysis lacks a out-of-sample validation. Of particular interest would be the validation of the Pareto shape parameter using the business as usual scenario. In fairness, the empirical validation would be beyond the scope of the paper, which aims at giving a qualitative account of the difference between the traditional and the proposed approach in climate-policy modelling. Furthermore, most of the concerned expressed here have been address in later work by \citein{Balistreri2013}.

\section{Conclusion}

The paper by \citein{Balistreri2012} gives a thorough account of the implications of the effects of climate policy via trade and intra-industry dynamics on carbon leakage. The exercise proposed by the authors gives strong quantitative reasons to not discount this effect as negligible. Despite some simplifying assumptions, the numerical analysis confirms the importance of carbon leakage in assessing climate policy. Furthermore, it highlights the potential role of carbon border adjustments in mitigating leakage as well as its reduced effectiveness when coalitions grow larger.

\newpage
% Bibliography
\nocite{*}
\pagenumbering{gobble} % stop page numbering
\printbibliography

\end{document}
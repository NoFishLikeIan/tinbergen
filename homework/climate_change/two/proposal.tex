\documentclass[american]{scrartcl}
    \usepackage{babel}
    \usepackage[utf8]{inputenc} 
    \usepackage{csquotes}
    \usepackage{amsmath}
    \usepackage{graphicx}    
    
    \setlength{\parindent}{0em}
    \setlength{\parskip}{0.5em}

    % Bibliography and citations
    \usepackage[bibencoding=utf8, style=apa]{biblatex}
    \bibliography{refs}

    
    \title{Does Uncertainty in Climate Collaboration Exacerbate the Run on Oil? A Cooperative Game Theory Perspective}

    % \subtitle{A critical essay on the existing literature}

    \author{Andrea Titton}
    

% Commands

\newcommand{\citein}[1]{\citeauthor{#1} (\citeyear{#1})}
\newcommand{\comment}[1]{\iffalse#1\fi}

\iffalse
    What (the research question is)? Does uncertainty in climate collaboration exacerbate the run on oil? A cooperative game theory perspective

    Why (is it an important question)? So far we only looked at climate policy uncertainty. But there might be further uncertainty deriving from a asynchronous implementation of climate policy 

    How (do we get to the answer)? Cooperative transferable utility games can model the issue. Use same model as van der Ploeg. Simulation in Julia. Estimation using Trump shock.
\fi
    
\begin{document}

% Title
\pagenumbering{gobble} % stop page numbering
\clearpage
\thispagestyle{empty}
\maketitle
\clearpage

\pagenumbering{arabic}

% Main

\section{Introduction}

Blablabla

\section{Current Literature}

The issue of stranded assets in face of uncertainty in climate policy has been thoroughly discussed. An overview on the matter is given by \citein{Ploeg2020}. The core idea is that fossil fuel producers face, on one hand, climate policy risk and, on the other hand, competition risk from renewable energy substitutes. Failure to commit to credible climate policies by governments might push fossil fuel producers to speed up investment and extraction in order to avoid the risk of stranded assets. This effect is due to the necessity of large intertemporal and intersectoral investments in fossil fuel extraction.

A game theoretical approach was proposed by \citein{Ploeg2020b}. The author puts forward a partial equilibrium model with oligopolistic fossil fuel producers, renewable energy producers and a second-best climate policy aiming at future cap on peak global warming. The analysis of the Nash equilibrium shows that, if there is a risk of credible enforcement of climate policy at some unknown future date anticipated by fossil fuel firms, fossil fuel extraction accelerates, thereby worsening global warming. The effect is exacerbated if the risk is higher and the expected date of the carbon cap enforcement is earlier. Furthermore, in a situation with no planned climate policy, a small risk of a future implementation on a global warming cap speeds up the rate of fossil fuel extraction, in particular under high competition in the fossil fuel market.

A complete economic analysis was carried out by \citein{Barnett2019}. The author relies on a general equilibrium model with oil extraction to illustrate this effect. Climate policy is in fact modeled as a Poisson regime shift in carbon pricing with arrival rate proportional to temperature (i.e. higher temperatures increase the probability of climate policy implementation). The model shows that introducing uncertainty in climate policy alters the subjective discount rate and induces a run on oil, as the likelihood of oil firms' assets becoming stranded increases. Furthermore, the author finds evidence of a ``carbon bubble'' caused by the underestimating of stranded assets risk.

The proposed research aims at combining the approaches of \citeauthor{Ploeg2020} and \citeauthor{Barnett2019} by giving an endogenous game theoretical formulation of the policy arrival rate.

\section{Theoretical framework}

The proposed model


\iffalse
    \section{Empirical approach}

    \section{Conclusion}

\fi

\newpage
% Bibliography
\nocite{*}
\pagenumbering{gobble} % stop page numbering
\printbibliography

\end{document}
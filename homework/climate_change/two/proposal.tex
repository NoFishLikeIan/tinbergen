\documentclass[american]{scrartcl}
    \usepackage{babel}
    \usepackage[utf8]{inputenc} 
    \usepackage{csquotes}
    \usepackage{amsmath}
    \usepackage{graphicx}    
    
    \setlength{\parindent}{0em}
    \setlength{\parskip}{0.5em}

    % Bibliography and citations
    \usepackage[bibencoding=utf8, style=apa]{biblatex}
    \bibliography{refs}

    
    \title{Would a Run on Oil lead to \\a Run on Climate Policy?}

    % \subtitle{A critical essay on the existing literature}

    \author{Andrea Titton}
    

% Commands

\newcommand{\citein}[1]{\citeauthor{#1} (\citeyear{#1})}
\newcommand{\comment}[1]{\iffalse#1\fi}

\iffalse
    What (the research question is)? Does uncertainty in climate collaboration exacerbate the run on oil? A cooperative game theory perspective

    Why (is it an important question)? So far we only looked at climate policy uncertainty. But there might be further uncertainty deriving from a asynchronous implementation of climate policy 

    How (do we get to the answer)? Cooperative transferable utility games can model the issue. Use same model as van der Ploeg. Simulation in Julia. Estimation using Trump shock.
\fi
    
\begin{document}

% Title
\maketitle

% Main

\section{Introduction}

A central topic in the climate economic literature has been that of stranded assets in face of uncertain climate policy. There are real concerns that the inability of policy makers to commit and enact swiftly climate policy might push fossil fuel firms to anticipate taxes or subsidies and trigger a run on oil. This anticipation mechanism might hinder the ability of governments to keep temperatures below a certain level. Previous research has focused on identifying this phenomenon and the mechanisms through which uncertainty might trigger a run on oil by making the policy decision fully exogenous. This approach, albeit fundamental in identifying the partial equilibrium dynamics, might obfuscate the interplay between oil firm's decision and climate policy. For example, oil exploration decision might put pressure on governments to step up climate policy (\cite{Ballew2019}) and this pressure might be heterogenous among countries.

To address this, I propose to relax the exogeneity assumption and develop a tractable game theoretical analysis of the interplay between uncertainty in climate policy and fossil fuel firms' decision making, with the aim of analyzing the role of international cooperation in exacerbating or mitigating the climate policy risk. Building on an existing private sector model of oil production and consumption, I will model government decisions by looking at the differential game generated by the interplay of oil extraction and exploration and climate policy.

Firstly, the model will be developed in a single country framework and I will derive the Feedback Nash equilibrium (FNE) of the game. Secondly, the model will be extended to a multi-country framework and the interaction between governments climate policies will be modeled as a cooperative differential game with transferable utility. Such a setup allows one to focus on the sources of the uncertainty, so far assumed to be exogenous, and potential feedback loops between uncertainty and firms' decisions. Finally, the production model will be calibrated in the same manner as \citein{Barnett2019}. The game theoretical model will be then empirically tested using the \citeauthor{Burck2020}.

\section{Literature review}

The issue of stranded assets in face of uncertainty in climate policy has been thoroughly discussed. An overview on the matter is given by \citein{Ploeg2020}. The core idea is that fossil fuel producers face, on one hand, climate policy risk and, on the other hand, competition risk from renewable energy substitutes. Failure to commit to credible climate policies by governments might push fossil fuel producers to speed up investment and extraction in order to avoid the risk of stranded assets. This effect is due to the necessity of large intertemporal and intersectoral investments in fossil fuel extraction.

A game theoretical approach, on the firms' side, was proposed by \citein{Ploeg2020b}. The author puts forward a partial equilibrium model with oligopolistic fossil fuel producers, renewable energy producers and a second-best climate policy aiming at future cap on peak global warming. The analysis of the Nash equilibrium shows that, if there is a risk of credible enforcement of climate policy at some unknown future date anticipated by fossil fuel firms, fossil fuel extraction accelerates, thereby worsening global warming. The effect is exacerbated if the risk is higher and the expected date of the carbon cap enforcement is earlier. Furthermore, in a situation with no planned climate policy, a small risk of a future implementation on a global warming cap speeds up the rate of fossil fuel extraction, in particular under high competition in the fossil fuel market.

A general equilibrium analysis was carried out by \citein{Barnett2019}. The author relies on a model with oil extraction to illustrate this effect. Climate policy is in fact modeled as a Poisson regime shift in carbon pricing with arrival rate proportional to temperature (i.e. higher temperatures increase the probability of climate policy implementation). The model shows that introducing uncertainty in climate policy alters the subjective discount rate and induces a run on oil, as the likelihood of oil firms' assets becoming stranded increases. Furthermore, the author finds evidence of a ``carbon bubble'' caused by the underestimating of stranded assets risk.

\section{Theoretical framework}

The proposed model will rely on a game theoretical framework to endogenise the policy arrival rate. This will be used to study the interplay between the policy formation and the equilibrium in the energy market, using the model developed by \citein{Barnett2019}.

\subsection{Production}

I assume that firms in the economy optimally choose investment levels, labour, and energy inputs in a perfectly competitive final good market. Energy is produced by ``green'' firms and oil firms. The former operate under perfect competition and have no negative externalities. The latter operate in a oligopolistic market, produce negative externalities, due to economic damages of temperature increases, and face investment costs of oil exploration. Furthermore, in absence of climate policy, demand for oil energy is structurally higher than that of green energy.

\subsection{Climate}

Atmospheric temperature above industrial levels is determined, in the model, by a carbon-climate component and a stochastic component (\cite{MacDougall2015}). A simple specification is chosen, as by \Citeauthor{Barnett2019}, in order to focus on long-run levels of temperature and keep the model tractable. Climate damage is modeled as an exponential shrinkage in temperature of income.

\subsection{Climate policy}

The policy maker can enact climate policy which yields a permanent jump in the energy input of oil. Previously, climate policy arrival has been modeled simply as a Poisson process with arrival rate increasing in temperature. I propose here to model the decision to enact climate policy as an optimization problem the policy maker faces.

The model will be first developed in a single country setting. The policy maker objective is to maximize total welfare by setting the arrival rate of climate policy. This assumption implies that the policy maker has partial control over the degree of uncertainty in policy implementation, which could reflect friction in the legislature procedure or political constraints. As in \citein{Barnett2019}, temperature increase reduces the value of both oil and green firms via damages. Government can decide to steepen the path of the arrival rate of policy, thus reducing uncertainty which would lower the final good output, or flattening it, thereby increasing uncertainty, boosting output and emissions. Such a system defines a differential game (\cite{Clemhout1979}) where oil firms determine oil explorations and the government determines the arrival rate of policy, thereby affecting the underline differential equation of temperature. Such a setup might highlight some of the short term rewards policy makers might be tempted to seize by keeping uncertainty high.

Solving the model in a single country setting then enables one to define a cooperative game with multiple governments facing the same differential game. The cooperative game can highlight the channel through which difference in climate policy preferences might induce further uncertainty in climate policy arrival. Furthermore, using cooperative games allows one to rely on established solution methods to identify possible transfer schemes that might reduce uncertainty and mitigate the problem.

I assume heterogeneity in countries' damage functions, energy and final good sector sizes. Such an heterogeneity leads to different governments having different incentives in increasing the arrival rate of climate policy. For illustration purposes, consider a two country sector both facing the same damage but one having a larger oil industry and another a larger green energy industry. The latter would want to reach climate policy via a path that keeps oil reserves stranded while the former would want for reserves to be extracted in order to obtain the benefit of an increased value of the oil sector. In such a setting, consensus can be reached via monetary transfers to convince the oil producing country to enact a different path of climate policy (\cite{Ju2006}).

\section{Empirical approach}

The model as presented in the previous section requires a two step calibration.

First, one needs to calibrate the model of production and energy sectors for each country. This has been done by \citein{Barnett2019} for the United States but the author's approach can be extended to other countries.

Second, one needs to calibrate the game theoretical model. The calibration can be done using the \citeauthor{Burck2020}. The index attempts to assess the commitment level of 57 major countries to climate policy. In the framework proposed here the main concern is that of relative commitment (i.e. for which countries is resolving uncertainty more beneficial than others), hence the index can be used to fully define the dynamics in the cooperative game.

\section{Conclusion}

The phenomenon of a run on oil might hinder the ability of governments to tackle climate change. Previous literature identified this mechanism, embedded it in economic models, and gave a clear policy implication, climate policy ought to be bold and swift. Here I propose to address the question that naturally arises from this claim: will a run on oil lead to a run on climate policy? Using a differential game, in a single country setting, and a cooperative game, in a multi-country framework, I will attempt to identify the feedback loop between the run of oil mechanisms and climate policy.

\newpage
% Bibliography
\nocite{*}
\pagenumbering{gobble} % stop page numbering
\printbibliography

\end{document}
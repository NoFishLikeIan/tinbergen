\documentclass{article}

\usepackage{fancyhdr}
\usepackage{extramarks}
\usepackage{amsmath}
\usepackage{amsthm}
\usepackage{amsfonts}
\usepackage{tikz}
\usepackage[plain]{algorithm}
\usepackage{algpseudocode}

\usetikzlibrary{automata,positioning}

%
% Basic Document Settings
%

\topmargin=-0.45in
\evensidemargin=0in
\oddsidemargin=0in
\textwidth=6.5in
\textheight=9.0in
\headsep=0.25in

\linespread{1.1}

\pagestyle{fancy}
\lhead{\hmwkAuthorName}
\chead{\hmwkTitle}
\rhead{\firstxmark}
\lfoot{\lastxmark}
\cfoot{\thepage}

\renewcommand\headrulewidth{0.4pt}
\renewcommand\footrulewidth{0.4pt}

\setlength\parindent{0pt}
\graphicspath{ {./images/} }

%
% Create Problem Sections
%

\newcommand{\enterProblemHeader}[1]{
    \nobreak\extramarks{}{Problem \arabic{#1} continued on next page\ldots}\nobreak{}
    \nobreak\extramarks{Problem \arabic{#1} (continued)}{Problem \arabic{#1} continued on next page\ldots}\nobreak{}
}

\newcommand{\exitProblemHeader}[1]{
    \nobreak\extramarks{Problem \arabic{#1} (continued)}{Problem \arabic{#1} continued on next page\ldots}\nobreak{}
    \stepcounter{#1}
    \nobreak\extramarks{Problem \arabic{#1}}{}\nobreak{}
}

\setcounter{secnumdepth}{0}
\newcounter{partCounter}
\newcounter{homeworkProblemCounter}
\setcounter{homeworkProblemCounter}{1}
\nobreak\extramarks{Problem \arabic{homeworkProblemCounter}}{}\nobreak{}

%
% Homework Problem Environment
%
% This environment takes an optional argument. When given, it will adjust the
% problem counter. This is useful for when the problems given for your
% assignment aren't sequential.
%
\newenvironment{homeworkProblem}[1][-1]{
    \ifnum#1>0
        \setcounter{homeworkProblemCounter}{#1}
    \fi
    \section{Problem \arabic{homeworkProblemCounter}}
    \setcounter{partCounter}{1}
    \enterProblemHeader{homeworkProblemCounter}
}{
    \exitProblemHeader{homeworkProblemCounter}
}

%
% Homework Details
%   - Title
%   - Due date
%   - Class
%   - Section/Time
%   - Instructor
%   - Author
%

\newcommand{\hmwkTitle}{Homework\ 5}
\newcommand{\hmwkDueDate}{October 11, 2019}
\newcommand{\hmwkClass}{Advanced Mathematics}
\newcommand{\hmwkAuthorName}{\textbf{Titton Andrea}}

%
% Title Page
%

\title{
    \vspace{2in}
    \textmd{\textbf{\hmwkClass:\ \hmwkTitle}}\\
    \normalsize\vspace{0.1in}\small{Due\ on\ \hmwkDueDate}\\
    \vspace{3in}
}

\author{\hmwkAuthorName}
\date{}

\renewcommand{\part}[1]{\textbf{\large Part \Alph{partCounter}}\stepcounter{partCounter}\\}

%
% Various Helper Commands
%

% Useful for algorithms
\newcommand{\alg}[1]{\textsc{\bfseries \footnotesize #1}}

% For derivatives
\newcommand{\deriv}[1]{\frac{\mathrm{d}}{\mathrm{d}x} (#1)}

% For partial derivatives
\newcommand{\pderiv}[2]{\frac{\partial}{\partial #1} (#2)}

% Integral dx
\newcommand{\dx}{\mathrm{d}x}

% Alias for the Solution section header
\newcommand{\solution}{\textbf{\large Solution}}

% Probability commands: Expectation, Variance, Covariance, Bias
\newcommand{\E}{\mathrm{E}}
\newcommand{\Var}{\mathrm{Var}}
\newcommand{\Cov}{\mathrm{Cov}}
\newcommand{\Bias}{\mathrm{Bias}}

\newcommand{\norm}[1]{\left\lVert#1\right\rVert}
\newcommand{\abs}[1]{\left\lvert#1\right\rvert}


\begin{document}

\maketitle

\pagebreak

\begin{homeworkProblem}

\subsection{a.}

If $J$ is in the form $J = \int_0^T L(t, x(t), \Dot{x}(t)) \ dt$, then by definition, the Euler-Lagrange equation of $J$ takes the form
\begin{equation}
   L_x (t, x, \Dot{x}) - \frac{d}{dt}\left(L_{\Dot{x}} (t, x, \Dot{x})\right) = 0
\end{equation} where $L_x$ is the partial derivative of $L$ with respect to $x$. \\
In this case then $L(t, x(t), \Dot{x}(t)) = (- \Dot{x}^2 + x^2)$, hence the corresponding Euler-Lagrange equation is
\begin{equation} \label{diff_eq}
    \begin{split}
        L_x (t, x, \Dot{x}) - \frac{d}{dt}\left(L_{\Dot{x}} (t, x, \Dot{x})\right) & = 0 \\
        2 x - \frac{d}{dt}(- 2 \Dot{x}) & = 0 \\
        x + \Dot{x} & = 0
    \end{split}
\end{equation}

Note that (\ref{diff_eq}) is a second order differential equation, hence we can find the root of its characteristic equation. Let $\Bar{z}$, be the aforementioned roots, then
\begin{equation}
    \begin{split}
        1 \cdot z^0 + 0 \cdot z^1 + 1 \cdot z^2 \dots& = 0 \\
        1 + z^2 & = 0 \\
        \Bar{z}_1 = i \ \Bar{z}_2 = -i
    \end{split}
\end{equation}

We then know that the family of solutions of (\ref{diff_eq}) takes the form
\begin{equation}
    \begin{split}
        x(t) & = C_1 e^{it} + C_2 e^{-it} \\ 
        &\text{by Euler's formula} \\
        x(t) &= C_1 cos(t) + C_2 sin(t)
    \end{split}
\end{equation}

We can then use the initial and end conditions, $x(0) = 0$ and $x(T) = 0$, to narrow the solution set. \\
$x(0) = 0$ implies 
\begin{equation}
    \begin{split}
        C_1 cos(0) + C_2 sin(0) &= 0 \\
        C_1 & = 0
    \end{split}
\end{equation}

and $x(T) = 0$ implies 
\begin{equation}
    C_2 sin(T) = 0
\end{equation}

hence either $C_2 = 0$ or $T = k \pi \land k \in \mathbb{Z}$. \\
Therefore, the solution to the Euler-Lagrange equation is $x(t) = 0$.

\subsection{b.}

Let the function form $x(t) = 0$, be $x_0$. It is trivial to show that $J(x_0) = 0$. \\

In order to prove this is not a maximum we need to find another function form in the same domain that gives a value of $J > 0$. Consider, for example, the function form
\begin{equation}
    x_n(t) = sin\left( \frac{2 \pi n}{T} t \right)
\end{equation}

If we evaluate $J$ at $x_n$ we obtain
\begin{equation} \label{bigger_j}
    J(x_n)=\int_{0}^{T}-\frac{4\pi^2 n^2}{T^2}\cos\left(\frac{2\pi n}{T}t\right)+\sin^2\left(\frac{2\pi n}{T}t\right)=\frac{T}{2}-\frac{2\pi^2 n^2}{T}=\frac{2\pi^2}{T}\left(\frac{T^2}{4\pi^2}-n^2\right)
\end{equation}

From (\ref{bigger_j}), you can see that if $\frac{T}{2\pi} > n$, $J(x_n) > 0$, hence $x_0$ is not a maximum.

\end{homeworkProblem}

\end{document}


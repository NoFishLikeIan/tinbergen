ss{article}

\usepackage{fancyhdr}
\usepackage{extramarks}
\usepackage{amsmath}
\usepackage{amsthm}
\usepackage{amsfonts}
\usepackage{tikz}
\usepackage[plain]{algorithm}
\usepackage{algpseudocode}

\usetikzlibrary{automata,positioning}

%
% Basic Document Settings
%

\topmargin=-0.45in
\evensidemargin=0in
\oddsidemargin=0in
\textwidth=6.5in
\textheight=9.0in
\headsep=0.25in

\linespread{1.1}

\pagestyle{fancy}
\lhead{\hmwkAuthorName}
\chead{\hmwkTitle}
\rhead{\firstxmark}
\lfoot{\lastxmark}
\cfoot{\thepage}

\renewcommand\headrulewidth{0.4pt}
\renewcommand\footrulewidth{0.4pt}

\setlength\parindent{0pt}
\graphicspath{ {./images/} }

%
% Create Problem Sections
%

\newcommand{\enterProblemHeader}[1]{
    \nobreak\extramarks{}{Problem \arabic{#1} continued on next page\ldots}\nobreak{}
    \nobreak\extramarks{Problem \arabic{#1} (continued)}{Problem \arabic{#1} continued on next page\ldots}\nobreak{}
}

\newcommand{\exitProblemHeader}[1]{
    \nobreak\extramarks{Problem \arabic{#1} (continued)}{Problem \arabic{#1} continued on next page\ldots}\nobreak{}
    \stepcounter{#1}
    \nobreak\extramarks{Problem \arabic{#1}}{}\nobreak{}
}

\setcounter{secnumdepth}{0}
\newcounter{partCounter}
\newcounter{homeworkProblemCounter}
\setcounter{homeworkProblemCounter}{1}
\nobreak\extramarks{Problem \arabic{homeworkProblemCounter}}{}\nobreak{}

%
% Homework Problem Environment
%
% This environment takes an optional argument. When given, it will adjust the
% problem counter. This is useful for when the problems given for your
% assignment aren't sequential.
%
\newenvironment{homeworkProblem}[1][-1]{
    \ifnum#1>0
        \setcounter{homeworkProblemCounter}{#1}
    \fi
    \section{Problem \arabic{homeworkProblemCounter}}
    \setcounter{partCounter}{1}
    \enterProblemHeader{homeworkProblemCounter}
}{
    \exitProblemHeader{homeworkProblemCounter}
}

%
% Homework Details
%   - Title
%   - Due date
%   - Class
%   - Section/Time
%   - Instructor
%   - Author
%

\newcommand{\hmwkTitle}{Homework\ 4}
\newcommand{\hmwkDueDate}{October 4, 2019}
\newcommand{\hmwkClass}{Advanced Mathematics}
\newcommand{\hmwkAuthorName}{\textbf{Titton Andrea}}

%
% Title Page
%

\title{
    \vspace{2in}
    \textmd{\textbf{\hmwkClass:\ \hmwkTitle}}\\
    \normalsize\vspace{0.1in}\small{Due\ on\ \hmwkDueDate}\\
    \vspace{3in}
}

\author{\hmwkAuthorName}
\date{}

\renewcommand{\part}[1]{\textbf{\large Part \Alph{partCounter}}\stepcounter{partCounter}\\}

%
% Various Helper Commands
%

% Useful for algorithms
\newcommand{\alg}[1]{\textsc{\bfseries \footnotesize #1}}

% For derivatives
\newcommand{\deriv}[1]{\frac{\mathrm{d}}{\mathrm{d}x} (#1)}

% For partial derivatives
\newcommand{\pderiv}[2]{\frac{\partial}{\partial #1} (#2)}

% Integral dx
\newcommand{\dx}{\mathrm{d}x}

% Alias for the Solution section header
\newcommand{\solution}{\textbf{\large Solution}}

% Probability commands: Expectation, Variance, Covariance, Bias
\newcommand{\E}{\mathrm{E}}
\newcommand{\Var}{\mathrm{Var}}
\newcommand{\Cov}{\mathrm{Cov}}
\newcommand{\Bias}{\mathrm{Bias}}

\newcommand{\norm}[1]{\left\lVert#1\right\rVert}
\newcommand{\abs}[1]{\left\lvert#1\right\rvert}


\begin{document}

\maketitle

\pagebreak

\begin{homeworkProblem}

\subsection{a.}

Let's take a look at $V = {x: x \in R, \norm{x} = 1}$. Set $V$ is bounded by definition. It is also closed because every sequence $x_n$ would have a limit in $V$ since $\forall x \in V \norm{x} = 1$. \\
We can than show that $f$ is continuous. Take $z, y: \norm{z - y} < \delta$, then $\forall  \ \epsilon > 0,  \ \exists \delta \ s.t$: 
\begin{equation} \label{ineq}
    \begin{split}
        \norm{f(z) - f(y)} & \leq \epsilon \\
        & \text{by using the definition of } f \\
        \norm{x^TAx - y^T A y} & \leq \epsilon \\
        \norm{(x^T- y^T) A (x-y)} & \leq \epsilon \\
        & \text{by triangular inequality} \\
        \norm{x-y} \norm{A} \norm{x-y} & \leq \epsilon \\
        & \text{using} \norm{x-y} = \delta \\
        \delta^2 \norm{A} \leq \epsilon
    \end{split}
\end{equation}

Hence by taking $\delta \leq \sqrt{\frac{\epsilon}{\norm{A}}}$ we satisfy (\ref{ineq}). 
Since $f$ is continuous on a compact space, by Weistrass' theorem, it must attain maximum and a minimum on V.

\subsection{b.}
Let $g(x) = \norm{x} - 1 = x^Tx - 1$, the critical points of $f$ must satisfy
\begin{equation} \label{eigen_space}
    \begin{split}
        \nabla f(x) & = \lambda \nabla g(x)\\
        Ax + x^T A  & = 2 \lambda x \\
        & \text{by symmetry of} A \\
        Ax & = \lambda x
    \end{split}
\end{equation}

Note that the equation $Ax = \lambda x$, characterizes the set of eigenvectors of $A$, with eigenvalues $\lambda$. We can then multiply the result of ($\ref{eigen_space}$) by $x^T$ on both sides we obtain,
\begin{equation} \label{res_eigen}
    \begin{split}
        Ax & = \lambda x \\
        x^T Ax & = \lambda x^T x \\
        f(x^*) = \lambda
    \end{split}
\end{equation}

Therefore, $f$ will be bounded in V by $\lambda$, where $\lambda$ are the eigenvalues of $A$. Note that we can trivially find maximum and minimum of the function by just taking the maximum and minimum eigenvalue respectively.

\subsection{c.}
We need to find a value $c > 0$, such that, for any matrix $M$ and vector $x$,
\begin{equation} \label{target}
    \norm{Mx} \leq c \norm{x}
\end{equation}

Let $y =\frac{x}{\norm{x}}$ and, by definition of vector norm, $x = \norm{x} y$. Note also that by construction $\norm{y}=1$. We can then rewrite (\ref{target}), for any matrix $M$ and vector $x$,
\begin{equation}
\begin{split}
    \norm{\norm{x}My} &\leq c \norm{\norm{x} y} \\
    & \text{since the norm of x is a scalar} \\
    \norm{x} \norm{My} & \leq \norm{x} c \norm{y} \\
    \norm{My} & \leq c \norm{y}
\end{split}
\end{equation}

Now let's expand the inequality with the definition of $\norm{v} = \sqrt{v^T v}$
\begin{equation}
    \begin{split}
        \norm{My} & \leq c \norm{y} \\
        \sqrt{(My)^T My} & \leq c \sqrt{y^T y} \\
        y^T M^T M y & \leq c^2 y^T y
    \end{split}
\end{equation}

Let now $A = M^T M$. This matrix is trivially squared and symmetric. \\

Then we can rewrite
\begin{equation} \label{redefined_target}
    y^T A y = c^2 y^T y
\end{equation}

By (\ref{res_eigen}), we know that a function of the form $y^T A y = \lambda y^T y$, has a maximum in its maximum eigenvalue, if $\norm{y}=1$ and $A$ is symmetric, which is the case in (\ref{redefined_target}). Hence by taking $c \geq \sqrt{\abs{\overline{\lambda}}}$, with $\overline{\lambda} = max\{\abs{\lambda}: det(M^T M - I \lambda) = 0\}$, equation (\ref{target}) is satisfied.

\end{homeworkProblem}

\begin{homeworkProblem}

Let's first transform the problem to standard coordinates. Let's define a mapping $\phi$
\begin{equation}
    \begin{split}
        & \phi(x) = (g_1(a), \dots, g_k(a)) \\
        & \phi(a) = 0
      \end{split}
\end{equation}

Note that, by construction, $D\phi(a)$ is invertible, hence there are neighbourhoods $U$ of $a$ and $M$ of 0 s.t. $\phi: U \xrightarrow{} M$ is a diffeomorphism.

Now let $\Tilde{g}_i(y) = g_i(\phi^{-1}(y)) \ \forall i$ and $\Tilde{f}(y) = f(\phi^{-1}(y))$, restricted to
\begin{equation}
    \Tilde{V} = M \cap \phi(V)
\end{equation}.

We can now using Taylor's expansion to rewrite $\Tilde{f}(y)$, noting that $\Tilde{f}(0) = f(a)$

\begin{equation} \label{taylor}
    \begin{split}
        \Tilde{f}(y) &= \Tilde{f}(0) + \sum_{i=1}^m \frac{\partial \Tilde{f}}{\partial y_i}(0) y_i + r(y)
    \end{split}
\end{equation}

Let's consider the term $ \sum_{i=1}^m \frac{\partial \Tilde{f}}{\partial y_i}(0) y_i$. \\
First take a vector $y^{(i)}$ s.t. $y^{(i)}_i = \epsilon$ and $y^{(i)}_j = 0$ for $i \neq j$, with $\epsilon < 0$ (this should be also close to zero to formally render $r(y)$ negligible). \\
We can now set $\lambda_i = \frac{\partial \Tilde{f}}{\partial y_i}(0)$ and by using $\nabla \Tilde{g}_i = e_i$, where $e_i$ is the $i$'th standard basis vector, we can rewrite
\begin{equation}
    \nabla \Tilde{f} = \begin{pmatrix} 
        \lambda_1\\ \vdots \\ \lambda_k
    \end{pmatrix} = \sum_{i}^k \lambda_i e_i = \sum_{i}^k \lambda_i \nabla \Tilde{g}_i(0)
\end{equation}

We can now find a relation between $\nabla f$ and $\nabla \Tilde{f}$ by taking the definition of $\Tilde{f}(\phi(x))$:
\begin{equation}
    \begin{split}
        \Tilde{f}(\phi(x)) & = f(x) \\
        Df(x) & = D\Tilde{f}(\phi(x))D\phi(x) \\
        \nabla f(x) & = D\phi(x)^T \nabla \Tilde{f}(y)\\
        & \text{evaluated at} \ a \\
        \nabla f(a) & = \sum_i^{k} \lambda_i \nabla g_i(a)
    \end{split}
\end{equation}

By assumption we know that 
\begin{equation}
    \lambda_i =  \frac{\partial \Tilde{f}}{\partial y_i}(0) > 0 \ \forall i
\end{equation}

By using this result in (\ref{taylor}),
\begin{equation}
    \begin{split}
        \frac{\partial \Tilde{f}}{\partial y_i}(0) = \Tilde{f(x)} - \Tilde{f(0)} < 0
    \end{split}
\end{equation}

Hence $\Tilde{f(0)}$ is a maximum, i.e. $\Tilde{f}$ takes a maximum in $0$. Then, by definition of $\Tilde{f}(y) = f(\phi^{-1}(y))$, $f$ must take a maximum in $\phi^{-1}(0) = a$
\end{homeworkProblem}

\end{document}


\documentclass[american]{scrartcl}
    \usepackage{babel}
    \usepackage[utf8]{inputenc} 
    \usepackage{csquotes}
    \usepackage{amsmath}
    \usepackage{graphicx} 
    \usepackage{csquotes}   
    
    \setlength{\parindent}{0em}
    \setlength{\parskip}{0.5em}

    % Bibliography and citations
    \usepackage[bibencoding=utf8, style=apa]{biblatex}
    \bibliography{refs}

    
    \title{Referee report: Optimal Fiscal Policy in the Presence of Declining Labor Share}

    \subtitle{Advanced Topics in Macro II}

    \author{Andrea Titton}
    

% Commands

\newcommand{\citein}[1]{\citeauthor{#1} (\citeyear{#1})}
\newcommand{\comment}[1]{\iffalse#1\fi}

    
\begin{document}

% Title

\maketitle

% Main

\section{Summary}

The paper by \citein{Atesagaoglu2020} seeks to determine the optimal allocation of the tax burden between capital and labour, under a regime of declining labour share of income. If one assumes the economy behaves as a neoclassical growth model with perfect competition and no frictions, in the long run, the tax burden ought to be placed entirely on the shoulders of labour income (\cite{Ramsey1928, Judd1985, Chamley1986}). The authors claim that the declining labour share is symptomatic of some underlying process that invalidates this result.

To address this, \citeauthor{Atesagaoglu2020} develop a standard neoclassical growth model with a representative agent. It is assumed there is a government that levies taxes on income, either capital or labour, to finance a stream of exogenous expenditures. The structure of the model allows incorporation of all potential mechanisms that might lead to a decline of labour share such that the authors can remain agnostic on the underlying process and give results contingent on the assumption made. The reasons behind a decline in the labour share can be grouped into two classes: either capital share of income is increasing due to automation or offshoring of labour-intensive production, or profit share is increasing due to market power effects, such as markups.

Another set of key assumptions the authors make to drive the model's dynamics concerns the taxing capacity of the government. First, optimal taxation in the neoclassical growth model requires the government to tax capital as much as possible in the first period, since this would be equivalent to a non-distortionary lump sum transfer. This is prevented by setting the initial tax rate on capital exogenously. Second, the model generates economic profits in the production sector due to markups which should be fully taxed in the optimum, as it would not affect the production decision. Such a taxation is usually not feasible for governments, hence the authors choose to set an exogenous upper bound to the tax on profits.


% FIXME: Explain the wedges better
These two assumptions are crucial in the model's result. The solution illustrates that, in the optimum, labour and capital taxes (or subsidies) are used to indirectly tax pure profits. The authors identify three wedges in the optimal labour tax. First, a ``Ramsey'' tax on labour income, which leverages the elasticity of consumption and leisure to extract tax revenue from labour. Second, a volume tax on labour income, which reduces economic profits by reducing labour supply. Third, a price subsidy on labour income, which boosts consumption, thereby reducing the price of the consumption good, hence the value of the untaxed profits. Similarly, the solution implies a volume wedge on capital, that is, an ex-ante increase in taxation of capital to reduce output, and an ex-post subsidy to capital to boost savings.

The model is then calibrated using results from the literature as source or indirect parametrization of target variables, such as risk free rates and hours worked. The calibration is done for two starting time points, namely 1980 and 2020. The main model validation is carried out by comparing net average product of capital in simulation and in the data.

The simulation exercise yields three main results. First, in place of increasing profits, optimal capital income taxes follow an increasing path while labour income taxes remain flat. Second, postponing optimal fiscal policy decision leads to higher optimal income taxes. Third, if the government is unable to correct monopolistic distortion, optimal taxes follow a similar path but settle at a lower steady state.

\section{Strengths}

First and foremost, the model proposed in this paper attempts to tackle a relevant pitfall of standard optimal taxation literature. To do so it employs a sensible modelling choice by assuming an underlying increase in either profits due to market power or in capital income. In this respect, it is, in theory, agnostic to the underlying process that drives the decline in the labour share and allows one to ask the question of optimal taxation under a broad set of assumptions. Furthermore, identifying the wedges in optimal taxation generated by the limits of the tax instruments is done clearly, rigorously, and in a way that allows one to build easily on the result. These wedges are given a strong intuitive underpinning that allows the model to serve well for both quantitative and qualitative analysis.

The model calibration is in line with common practice in macroeconomic literature. In particular the authors take care of using different benchmark calibrations when there is disagreement in the literature on a variable, in this case the elasticity of substitution between intermediate goods due to different estimation of pure profits.

\section{Criticisms}

From a theoretical point of view, the choice of modelling government expenditure as an exogenous stream of obligations might be incorrect. Such a choice is very common in the neoclassical growth model and would be suitable in case one were looking exclusively at how the government should finance such expenditure. But in this case, government expenditure might have strong implications for the dynamics of the labour share of income, which are crucial in determining the optimal taxation. For example, \citein{Facchini2016} find that the labour share of income affects government expenditure via labour costs. Such a feedback loop should be explicitly modeled.

Another assumption that should be further justified and the consequences of which should be better clarified, is that of the inability of governments to tax pure profits. A lot of the model dynamics are driven by this government limitation since the tax wedges later identified serve the goal of indirectly taxing pure profits. \citeauthor{Atesagaoglu2020} claim that,

\begin{displayquote}[p. 12]
    there may be a number of reasons in reality for why taxing pure profits at 100\% may not be such a great idea.
\end{displayquote}

This is probably true but the authors should elaborate on why governments might be unwilling or unable to tax pure profits. For example, \citein{Huizinga1997} address this by stating that,


\begin{displayquote}[p. 156]
    In practice, complete profit taxation is rare for various reasons. First, the tax authority may have difficulty distinguishing between pure profits and the return to capital investment, in which case full profit taxation is possible but not desirable. Second, full profit taxation may be impossible, if there are institutional or legal restraints. Finally, significant profit taxation is precluded by tax competition, if firms are internationally mobile, and if profits or rents are firm-specific rather than location-specific.
\end{displayquote}

and proceed to justify why such limitations do not have general equilibrium implications in the context of their model. I believe a similar approach should be taken in the paper in exam.

Finally, the authors decide to calibrate the disutility of hours worked to target the labour supply to be $8/24$. Such a choice implicitly assumes that, first, the supply of labour arises entirely due to an unconstrained endogenous process, and, second, that the heterogeneity in hours worked is not relevant in the context of the model. Again, such an assumption might be justified in a purely partial equilibrium model, where one wants to remain agnostic on the underlying process of labour formation, but in this case the dynamics and causes of the labour share decline are the foundations of the authors' claim. The regulatory causes behind and inherent heterogeneity of hours worked in the economy might hinder the assumptions on the drivers of the decline labour share, thereby invalidating the authors' calibrated model. As the authors state,

\begin{displayquote}[p. 3][]
    We focus on a representative agent framework intentionally in order to avoid issues of inequality and redistribution.
\end{displayquote}

hence a reasonable approach would be to introduce a robustness test by perturbation of the disutility of hours worked parameter.

\section{Suggestions}

First and foremost, the authors should employ a set of robustness checks in their calibration procedure. On the one hand, parameters robustness should be established by adding confidence intervals to the parameters and analyzing the out of sample data fit within the confidence intervals. On the other hand, model robustness should be established by analyzing the qualitative results for different specifications of optimization functions. In particular, the results of relaxing separability in the utility function should be reported.

Another aspect of the model that I think requires further elaboration is that of the counterfactual analysis on the rise in capital share (Section 5.5). The authors discount the analysis since it produces an average product of capital path that contradicts the empirical evidence. I believe the authors should elaborate on the drivers behind this result which seems prima facie entirely assumption driven.

Finally, the authors should expand the analysis and calibration in case of a government unable to correct monopolistic distortions (Section 4 and Section 5.4). Of particular interest is the intuition behind the qualitative result that optimal taxes in this case are smaller than in the benchmark case. Judging from the preliminary state of these section, it seems that the author already intend on doing this.

\newpage
% Bibliography
\nocite{*}
\pagenumbering{gobble} % stop page numbering
\printbibliography

\end{document}
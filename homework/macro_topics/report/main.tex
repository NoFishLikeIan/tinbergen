\documentclass[american]{scrartcl}
    \usepackage{babel}
    \usepackage[utf8]{inputenc} 
    \usepackage{csquotes}
    \usepackage{amsmath}
    \usepackage{graphicx}    
    
    \setlength{\parindent}{0em}
    \setlength{\parskip}{0.5em}

    % Bibliography and citations
    \usepackage[bibencoding=utf8, style=apa]{biblatex}
    \bibliography{refs}

    
    \title{Referee report: Optimal Fiscal Policy in the Presence of Declining Labor Share}

    \subtitle{Advanced Topics in Macro II}

    \author{Andrea Titton}
    

% Commands

\newcommand{\citein}[1]{\citeauthor{#1} (\citeyear{#1})}
\newcommand{\comment}[1]{\iffalse#1\fi}

    
\begin{document}

% Title

\maketitle

% Main

\section{Summary}

The paper by \citein{Atesagaoglu2020} seeks to determine the optimal allocation of the tax burden between capital and labour, under a regime of declining labour share of income. If one assumes the economy behaves as a neoclassical growth model with perfect competition and no frictions, in the long run, the tax burden ought to be placed entirely on the shoulders of labour income (\cite{Ramsey1928, Judd1985, Chamley1986}). The authors, correctly in my view, claim that the declining labour share is symptomatic of some underling process that invalidates this result.

To address this \citeauthor{Atesagaoglu2020} develop a standard neoclassical growth model with a representative agent. It is assumed there is a government that levies taxes on income, either capital or labour, to finance a stream of exogenous expenditures. The structure of the model allows to incorporate all potential mechanisms that might lead to a decline of labour share. In this way the authors can remain agnostic on the underling process and give results contingent on the assumption made. Such assumption can fall into two classes: either capital share of income is increasing, due to automation or offshoring of labour-intensive production, or profit share are increasing due to market power effects, such as markups.

Another set of key assumptions the authors make to drive the model's dynamics concerns the taxing capacity of the government. First, optimal taxation in the neoclassical growth model requires the government to tax capital as much as possible in the first period, since this would be equivalent to a non-distortionary lump sum transfer. This is prevented by setting the initial tax rate on capital exogenously. Second, the model generates economic profits in the production sector due to markups which should be fully taxed in the optimum, as it would not affect the production decision. Such a taxation is usually not feasible for governments hence the author choose to set an exogenous upper bound to the tax on profits.


% FIXME: Explain the wedges better
These two assumptions are crucial in the model's result. The solution illustrates that in the optimum labour and capital taxes and subsidies are used to indirectly tax pure profits. The authors identify three wedges in the optimal labour tax. First a ``Ramsey'' tax on labour income, which leverages the elasticity of consumption and leisure to extract tax revenue from labour. Second a volume tax on labour income, which aims at reducing economic profits by reducing labour supply. Third a price subsidy on labour income, which boosts consumption thereby reducing the price of the consumption good hence the value of the untaxed profits. Similarly, the solution implies a volume wedge on capital, that is, an increase in taxation ex-ante of capital to reduce output, and an ex-post subsidy to capital to boost savings.

The model is then calibrated using results from the literature as source or indirect parametrization of target variables, such as risk free rates and hours worked. The calibration is done for two starting time points, namely 1980 and 2020.

% TODO: Finish the summary

\newpage
\section{Strengths}

First and foremost, the model proposed in this paper attempts to tackle a relevant pitfall of standard optimal taxation literature. In order to do so it employs a sensible modelling choice by assuming an underling profit increase due to market power or a capital income increase. In this respect, it is, in theory, agnostic to the underline process that drives the decline in the labour share and allows one to ask the question of optimal taxation under a broad set of assumptions. Furthermore, identifying the wedges in optimal taxation generated by the limits of the tax instruments is done clearly, rigorously, and in a way that allows one to build easily on the result. These wedges are given a strong intuitive underpinning that allows the model to serve well for both quantitative and qualitative analysis.

The model calibration is in line with the common practice in macroeconomic literature. In particular the authors take care of using different benchmark calibration when there is disagreement in the literature on a variable, in this case the elasticity of substitution between intermediate goods due to different estimation of pure profits.



\newpage
\section{Criticalities}

\newpage
\section{Suggestions}


\newpage
% Bibliography
\nocite{*}
\pagenumbering{gobble} % stop page numbering
\printbibliography

\end{document}